% $Header$

\documentclass[aspectratio=169,compress]{beamer}

\mode<presentation>
{
  \usetheme[sectionpage=progressbar,numbering=none,progressbar=frametitle,block=fill]{metropolis}
  % \usetheme{Rochester}

  \usecolortheme[cautious,snowy]{owl}
  % \usecolortheme{crane}
  % \usecolortheme{beaver}

  \setbeamercovered{transparent}
  % or whatever (possibly just delete it)
}

% \setbeamertemplate{navigation symbols}{}
% \setbeamertemplate{frame footer}{My custom footer}

\definecolor{OwlGray}{HTML}{808080}

\setbeamercolor{progress bar}{fg=OwlYellow, bg=OwlYellow!30}
\setbeamercolor{alerted text}{fg=OwlYellow, bg=OwlYellow!30}
% \setbeamercolor{frametitle}{bg=OwlGreen}
\setbeamercolor{framesubtitle}{fg=OwlGray, bg=normal text.bg}


% \useoutertheme[subsection=false]{miniframes}

\usepackage[english]{babel}
% or whatever

% \usepackage[latin1]{inputenc}
% or whatever

\usepackage[T1]{} % +{fontenc}
% Or whatever. Note that the encoding and the font should match. If T1
% does not look nice, try deleting the line with the fontenc..

\usepackage{metropolis_framesubtitle}
\usepackage[most]{tcolorbox}
\usepackage{appendixnumberbeamer}
\usepackage{lipsum,tikz}
    \usetikzlibrary{shapes.geometric,calc,arrows.meta,bending,positioning}
\usepackage{xspace}
\usepackage{subcaption}
\usepackage{amsmath}
\usepackage[mathrm=sym]{unicode-math}
\usepackage{mathtools}

\usefonttheme[onlymath]{serif}
% \setmathfont{Fira Math}

% HL
\newcommand{\HL}{\textsc{Hiding Leaders}\xspace}
\newcommand{\HLshort}{\textsf{HL}\xspace}
\newcommand{\HLdeg}{\textsf{HL}$_{deg}$\xspace}
% params
\newcommand{\leaddeg}{$\lambda$}
\newcommand{\budget}{$b$}
\newcommand{\safetymarg}{$d$}
\newcommand{\leadnum}{$|L|$}
% complexity
\newcommand{\Po}{\textsf{P}\xspace}
\newcommand{\NP}{\textsf{NP}\xspace}
\newcommand{\NPh}{\textsf{NP}-hard\xspace}
\newcommand{\NPc}{\textsf{NP}-complete\xspace}
\newcommand{\FPT}{\textsf{FPT}\xspace}
\newcommand{\XP}{\textsf{XP}\xspace}
\newcommand{\W}{\textsf{W}\xspace}
\newcommand{\Wone}{\textsf{W}[1]\xspace}
\newcommand{\Wh}{\textsf{W}[1]-hard\xspace}
\newcommand{\pNP}{\textsf{para-NP}\xspace}
\newcommand{\pNPh}{\textsf{para-NP}-hard\xspace}

% \theoremstyle{plain}
\newtheorem*{theorem*}{Theorem}


% -----------------------------------------------------------------------------------------------
% -----------------------------------------------------------------------------------------------
\title[Short Paper Title] % (optional, use only with long paper titles)
{Hiding Leaders in Covert Networks:\linebreak
A Computational Complexity Perspective}

% \subtitle
% {Bachelor's thesis presentation} % (optional)

\author[Author, Another] % (optional, use only with lots of authors)
{Patrik~Drbal\\[\baselineskip]Supervisor: Ing. Šimon Schierreich\\[\baselineskip]}
% - Use the \inst{?} command only if the authors have different
%   affiliation.

\date[Short Occasion] % (optional)
{June 20, 2023}

% If you have a file called "university-logo-filename.xxx", where xxx
% is a graphic format that can be processed by latex or pdflatex,
% resp., then you can add a logo as follows:

% \pgfdeclareimage[height=0.5cm]{university-logo}{cvut_logo}
% \logo{\pgfuseimage{university-logo}}



% Delete this, if you do not want the table of contents to pop up at
% the beginning of each subsection:
% \AtBeginSubsection[]
% {
%   \begin{frame}<beamer>{Outline}
%     \tableofcontents[currentsection,currentsubsection]
%   \end{frame}
% }


% If you wish to uncover everything in a step-wise fashion, uncomment
% the following command: 
%\beamerdefaultoverlayspecification{<+->}


% -----------------------------------------------------------------------------------------------
% -----------------------------------------------------------------------------------------------
\begin{document}

% FRAME - Title
\begin{frame}
  \titlepage
\end{frame}

% FRAME - Outline
% \begin{frame}{Outline}
%   \tableofcontents[pausesections]
%   % You might wish to add the option [pausesections]
% \end{frame}

\section{Introduction}

% FRAME
\begin{frame}{Motivation}
    \begin{itemize}
        \item Help to deepen the understanding of hiding in networks
        \item Help to make identifying key actors in criminal organizations easier
    \end{itemize}
\end{frame}

% FRAME
\begin{frame}{Problem Definition}{}
    \HL
    \only<-3>{(\HLshort)}%
    \only<4->{(\alert{\HLdeg})}

    \uncover<+->{
        \uncover<+->{
            \emph{Instance}: $(G, L, b, c, d)$
            \begin{itemize}
                \item $G = (V,E)$ is a network
                \item $L \subseteq V$ are leaders, $F = V \smallsetminus L$ are followers
                \item $b$ is a budget
                \item \only<-3>{$c : G \times V \rightarrow \mathbb{R}$ is a centrality measure}
                \only<4->{\alert{$c = c_{deg}$, $c_{deg}(G, v) = \deg(v)$}}
                \item $d$ is a safety margin
            \end{itemize}
        }

        \uncover<+->{
            \emph{Question}: Do exist $W \subseteq F \times F$ and $F' \subseteq F$?
            \begin{itemize}
                \item $|W| \leq b$
                \item $|F'| \geq d$
                \item \only<-3>{$\forall_{f' \in F'} \forall_{\ell \in L} \colon c(G', f') \geq c(G', \ell)$, where $G' = (V, E \cup W)$}
                \only<4->{\alert{$\forall_{f' \in F'} \colon \deg_{G'}(f') \geq \max\limits_{\forall_{\ell \in L}}\deg_{G'}(\ell)$, where $G' = (V, E \cup W)$}}
            \end{itemize}
        }
    }
\end{frame}

% FRAME
\begin{frame}{Sample Instance}
    \begin{figure}[t!]
    \centering
    \begin{tikzpicture}[
        node distance={25mm}, every node/.style={scale=1},
        ampersand replacement=\&,
        legend line/.pic={
          \draw[yshift=.5ex,thick,#1] (0, 0) -- (0.6, 0);
        },
        legend lineDashed/.pic={
          \draw[yshift=.5ex,thick,dashed,#1] (0, 0) -- (0.6, 0);
        },
        legend area/.pic={
          \draw[yshift=.5ex,fill=#1] (0, -0.2) rectangle (0.6, 0.2);
        }
      ]
        \tikzstyle{square} = [regular polygon, regular polygon sides=4]
        \tikzstyle{leader} = [square, fill=blue!50]
        \tikzstyle{edgeL} = [color=blue!50, thick]
        \tikzstyle{follower} = [circle, draw=black]
        \tikzstyle{followerF} = [circle, fill=green!40]
        \tikzstyle{edgeW} = [draw=green!40, thick]
        \tikzstyle{edgePotential} = [dashed, draw=green!40, thick]

        \begin{scope}[local bounding box=L]
            \node[leader] (l1) at (0,0) {$\ell_1$};
            \node[leader] (l2) [right of = l1] {$\ell_2$};
            \node[leader] (l3) [right of = l2] {$\ell_3$};
            \node[follower] (f1) [below right=15mm and 6mm of l1] {$f_1$};
            \only<1-3>{\node[follower] (f2) [below left=15mm and 6mm of l3] {$f_2$};}
            \only<4->{\node[followerF] (f2) [below left=15mm and 6mm of l3] {$f_2$};}
            \node[follower] (f3) [below of = f1] {$f_3$};
            \only<1-3>{\node[follower] (f4) [below of = f2] {$f_4$};}
            \only<4->{\node[followerF] (f4) [below of = f2] {$f_4$};}

            \only<1>{
                \draw (l1) to [out=45,in=135,looseness=0.5] (l3);
                \draw (l2)--(l3);
                \draw (l3)--(f2);
            }
            \only<2->{
                \draw[edgeL] (l1) to [out=45,in=135,looseness=0.5] (l3);
                \draw[edgeL] (l2)--(l3);
                \draw[edgeL] (l3)--(f2);
            }

            \draw (l2)--(f2);
            \draw (f2)--(f4);
            \draw (f3)--(f4);

            \only<3>{
                \draw[edgePotential] (f1)--(f2);
                \draw[edgePotential] (f1)--(f3);
                \draw[edgePotential] (f1)--(f4);
                \draw[edgePotential] (f2)--(f3);
            }
            \only<4->{\draw[edgeW] (f1)--(f4);}
        \end{scope}

        % legend
        \begin{scope}[shift={(L.center)}]
            \only<1-3>{
                \small
                \matrix [column sep=2mm, draw, above right] at (current bounding box.south east) {
                    \pic {legend line=white}; \& \node {$b=1$, $c=c_{deg}$, $d=2$}; \& \&\\
                };
            }
            \only<4->{
                \small
                \matrix [column sep=2mm, draw, above right] at (current bounding box.south east) {
                    \pic {legend line=white}; \& \node {$b=1$, $c=c_{deg}$, $d=2$}; \& \&\\
                    \pic {legend line=green!40}; \& \node {$W$}; \& \&\\
                    \pic {legend area=green!40}; \& \node {$F'$}; \& \&\\
                };
            }
        \end{scope}
    \end{tikzpicture}
\end{figure}
\end{frame}

% FRAME
\begin{frame}{Goals}
    \begin{itemize}
        \item Survey the computational complexity of the problem variants studied in the literature
        \item Derive new complexity and algorithmic results for various parameters
        with the use of the framework of parameterized complexity
    \end{itemize}
\end{frame}

% FRAME
\begin{frame}{Parameterized complexity classes}
    \begin{columns}
        \begin{column}{0.39\textwidth}
            \begin{itemize}
                \item $\FPT: f(k) \cdot |(x,k)|^c$
                \item $\XP: f(k) \cdot |(x,k)|^{g(k)}$
            \end{itemize}

            \begin{center}
                \line(1,0){\textwidth/3}
            \end{center}

            \begin{itemize}
                \item \Wh $\Rightarrow$ not in \FPT
                \item \pNPh $\Rightarrow$ not in \XP
            \end{itemize}
        \end{column}
    
        \begin{column}{0.69\textwidth}
            % \begin{figure}[t]
    \centering
    \begin{tikzpicture}[remember picture, overlay, yshift=-3cm]
        %% Horizontal bar
        \draw[very thick] (3,0) -- (-3,0);
        % FPT
        \draw (-2,0) parabola bend (0,2) (2,0);
        \node at (0,1) {\FPT};
        % W[1]
        \draw (-2.5,0) parabola bend (0,3) (2.5,0);
        \node at (0,2.5) {\textsf{W}[1]};
        % XP
        \draw[rounded corners] (-3,0) .. controls (-1,4) and (5,9) .. (3,0);
        \node at (2, 4) {\XP};
        % para-NP
        \draw[rounded corners] (3,0) .. controls (1,4) and (-5,9) .. (-3,0);
        \node at (-2, 4) {\pNP};
    \end{tikzpicture}
    % \caption
    % [Relations among the parameterized complexity classes]
    % {
    %     Relations among the parameterized complexity classes.
    %     Inspired by the depiction from Flum and Grohe \cite[p.~97]{Flum2006}.
    % }
    % \label{fig:complexityClasses}
% \end{figure}
  
        \end{column}
      \end{columns}
\end{frame}

\section{Contribution}

% FRAME
\begin{frame}{Literature Review}
    \begin{itemize}
        \item Literature results summarized and described in the language of parameterized complexity theory
        \item Pointed out that there are two definitions of the problem
        \begin{itemize}
            \item Proof of \NP-hardness of \HLdeg reviewed for our definition 
        \end{itemize}
    \end{itemize}
\end{frame}

% ----------------------------------------------------
\subsection
{Own results}

% FRAME
\begin{frame}{Own Results}
    \begin{figure}[t]
    \centering
    \begin{tikzpicture}[
      node distance=19mm,
      ampersand replacement=\&,
      legend line/.pic={
        \draw[yshift=.5ex,thick,#1] (0, 0) -- (0.6, 0);
      },
      legend area/.pic={
        \draw[yshift=.5ex,fill=#1] (0, -0.2) rectangle (0.6, 0.2);
      }
    ]
      \tikzstyle{every node} = [minimum height=0.7cm,minimum width=1cm,scale=1.0];
      \tikzstyle{result} = [draw, very thick];
      \tikzstyle{corollary} = [draw,dashed];
      \tikzstyle{pNPh} = [fill=red!30];
      \tikzstyle{Wh} = [fill=violet!30];
      \tikzstyle{WhREAL} = [fill=orange!30];
      \tikzstyle{XP} = [fill=green!30];

      \tikzstyle{edgeXP} = [color=orange,-{Stealth[length=3mm]}]
      \tikzstyle{edgeW} = [color=violet,-{Stealth[length=3mm]}]

      \tikzstyle{dummy} = [color=white, text=white]

      % pairs
      \node[corollary,WhREAL,label={0:\small{\tt C\,4.7}}] (LnB) at (0,0) {\leadnum, \budget};
      \node[corollary,WhREAL,label={0:\small{\tt C\,4.7}}] (LnD) [right =of LnB] {\leadnum, \safetymarg};
      \node[corollary,WhREAL,label={0:\small{\tt C\,4.7}}] (BD) [right=of LnD] {\budget, \safetymarg};
      % singletons
      \node[result,WhREAL,label={0:\small{\tt T\,4.9}}] (B) [above left=18mm and 2mm of LnD] {\budget};
      \node[result,WhREAL,label={0:\small{\tt T\,4.10}}] (D) [above right=18mm and 2mm of LnD] {\safetymarg};
      \node[corollary,pNPh,label={0:\small{\tt C\,4.8}}] (Ln) [left=of B] {\leadnum};
      \node[XP,label={0:\small{\tt known}}] (Ld) [right=of D] {\leaddeg};
      % triplets
      \node[result,WhREAL,label={0:\small{\tt R\,4.6}}] (LnBD) [below=of LnD] {\leadnum, \budget, \safetymarg};
      % edges
      \draw[->,edgeW] (LnBD.45) -- (LnB.315);
      \draw[->,edgeW] (LnBD.45) -- (LnD.315);
      \draw[->,edgeW] (LnBD.45) -- (BD.315);
      \draw[->,edgeW] (BD.45) -- (B.315);
      \draw[->,edgeW] (BD.45) -- (D.315);
      \draw[->,edgeXP] (B.225) -- (LnB.135);
      \draw[->,edgeXP] (D.225) -- (LnD.135);
      \draw[->,edgeXP] (D.225) -- (BD.135);    
      \draw[->,edgeXP] (BD.225) -- (LnBD.135);
      % legend
      \small
      \matrix [column sep=2mm, draw, anchor=west, xshift=1.8cm, yshift=0.6cm] at (Ld |- LnBD){
        \pic {legend area=green!30}; \& \node {\XP}; \& \&\\
        \pic {legend area=orange!30}; \& \node {\Wh, \XP}; \& \&\\
        \pic {legend area=red!30}; \& \node {\pNPh}; \& \&\\
        \pic {legend line=orange}; \& \node {\XP}; \& \&\\
        \pic {legend line=violet}; \& \node {\Wh}; \& \&\\
      };
    \end{tikzpicture}
    \label{fig:complexityPicture}
\end{figure}

\end{frame}

% FRAME
\begin{frame}{Own Results}{Theorem}
    \begin{center}
        \tcbox[enhanced, skin=enhancedmiddle, arc=0pt, outer arc=0pt, frame hidden, colback=OwlGray!10, boxsep=0mm] {
            \HL parameterized by $b + d$ is \Wh even if $|L| = 1$
        }
    \end{center}
    \begin{center}
        \tcbox[enhanced, skin=enhancedmiddle, arc=0pt, outer arc=0pt, frame hidden, colback=OwlGray!10, boxsep=0mm] {
            \HL parameterized by $|L| + b + d$ is \Wh
        }
    \end{center}

    \begin{itemize}
        \item Proved by showing a parameterized reduction from $k$-\textsc{Clique} on regular graphs
        \begin{itemize}
            \item $k$-\textsc{Clique} instance $(G, k)$, where $G$ is a $r$-regular graph
            \item \HL instance $(H, \{\ell\}, \frac{k\cdot(k-1)}{2}, c_{deg}, k)$, where $H$ is constructed from $G$
            \item Solving \HL in $H$ corresponds to solving $k$-\textsc{Clique} in $G$
        \end{itemize}
        \item Enhanced version of the proof from Waniek et al.
    \end{itemize}
\end{frame}

% FRAME
\begin{frame}{Sample Construction}
    \begin{figure}[t!]
  \centering
  \begin{tikzpicture}[
      node distance={25mm}, every node/.style={scale=0.75},
      ampersand replacement=\&,
      legend line/.pic={
        \draw[yshift=.5ex,thick,#1] (0, 0) -- (0.6, 0);
      },
      legend lineDashed/.pic={
        \draw[yshift=.5ex,thick,dashed,#1] (0, 0) -- (0.6, 0);
      },
      legend area/.pic={
        \draw[yshift=.5ex,fill=#1] (0, -0.2) rectangle (0.6, 0.2);
      }
    ]
    \tikzstyle{followerG} = [circle, draw=black]
      \tikzstyle{edgeG} = [dashed, draw=green!40, thick]

      \tikzstyle{edgeCompl} = []

      \tikzstyle{followerF2} = [circle, fill=red!50]
      \tikzstyle{edgeF2} = [draw=red!50]

      \tikzstyle{square} = [regular polygon, regular polygon sides=4]
      \tikzstyle{leader} = [square, fill=blue!50, label={[text=blue!50]above:$\ell$}]
      \tikzstyle{edgel} = [color=blue!50]
      \tikzstyle{dummy} = [color=white, label={[text=white]above:$\ell$}]

      \only<1>{
        \node[followerG, text=white] (g1) at (0,0) {$f_{1_1}$};
        \node[followerG, text=white] (g2) [right of = g1] {$f_{1_2}$};
        \node[followerG, text=white] (g3) [below of = g1] {$f_{1_3}$};
        \node[followerG, text=white] (g4) [right of = g3] {$f_{1_4}$};
        \node[followerG, text=white] (g5) [below of = g3] {$f_{1_5}$};
        \node[followerG, text=white] (g6) [right of = g5] {$f_{1_6}$};
      }
      \only<2->{
        \node[followerG] (g1) at (0,0) {$f_{1_1}$};
        \node[followerG] (g2) [right of = g1] {$f_{1_2}$};
        \node[followerG] (g3) [below of = g1] {$f_{1_3}$};
        \node[followerG] (g4) [right of = g3] {$f_{1_4}$};
        \node[followerG] (g5) [below of = g3] {$f_{1_5}$};
        \node[followerG] (g6) [right of = g5] {$f_{1_6}$};
      }

      % dummy leader and partners to prevent graph size
      \only<1-2>{
        \node[leader, dummy] (l) [right of = g4] {};
        \draw[edgel, dummy] (l)--(g4);
        \draw[edgel, dummy] (l)--(g6);
        \draw[edgel, dummy] (l) to [out=45,in=90] (g1);
        \draw[edgel, dummy] (l) to [out=315,in=270] (g5);
      }
      \only<1-3>{
        \node[followerF2, label={[text=white]above:$f_{2_1}$}, dummy] (f2) [above of = g2] {};
        \node[followerF2, label={[text=white]above:$f_{2_2}$}, dummy] (f3) [left of = g3] {};
        \draw[edgeF2, dummy] (f2)--(g2);
        \draw[edgeF2, dummy] (f3)--(g3);
      }

      \only<1>{
        \draw[edgeCompl] (g1)--(g2);
        \draw[edgeCompl] (g5)--(g6);
        \draw[edgeCompl] (g1)--(g3);
        \draw[edgeCompl] (g3)--(g5);
        \draw[edgeCompl] (g2)--(g4);
        \draw[edgeCompl] (g4)--(g6);
        \draw[edgeCompl] (g3)--(g2);
        \draw[edgeCompl] (g3)--(g6);
        \draw[edgeCompl] (g4)--(g1);
        \draw[edgeCompl] (g4)--(g5);
        \draw[edgeCompl] (g1) to [out=180,in=180,looseness=0.5] (g5);
        \draw[edgeCompl] (g2) to [out=0,in=0,looseness=0.5] (g6);
      }
      \only<2->{
        \draw[edgeG] (g1)--(g2);
        \draw[edgeG] (g5)--(g6);
        \draw[edgeG] (g1)--(g3);
        \draw[edgeG] (g3)--(g5);
        \draw[edgeG] (g2)--(g4);
        \draw[edgeG] (g4)--(g6);
        \draw[edgeG] (g3)--(g2);
        \draw[edgeG] (g3)--(g6);
        \draw[edgeG] (g4)--(g1);
        \draw[edgeG] (g4)--(g5);
        \draw[edgeG] (g1) to [out=180,in=180,looseness=0.5] (g5);
        \draw[edgeG] (g2) to [out=0,in=0,looseness=0.5] (g6);
        % complement
        \draw[edgeCompl] (g1)--(g6);
        \draw[edgeCompl] (g2)--(g5);
        \draw[edgeCompl] (g3)--(g4);
      }
      % leader
      \only<3->{
        \node[leader] (l) [right of = g4] {};
        \draw[edgel] (l)--(g4);
        \draw[edgel] (l)--(g6);
        \draw[edgel] (l) to [out=45,in=90] (g1);
        \draw[edgel] (l) to [out=315,in=270] (g5);
      }
      % partners
      \only<4->{
        \node[followerF2, label={[text=red!50]above:$f_{2_1}$}] (f2) [above of = g2] {};
        \node[followerF2, label={[text=red!50]above:$f_{2_2}$}] (f3) [left of = g3] {};
        \draw[edgeF2] (f2)--(g2);
        \draw[edgeF2] (f3)--(g3);
      }
      % legend
      \only<1-2>{
          \small
          \matrix [column sep=2mm, draw, above right, yshift=2cm] at (current bounding box.south east) {
            \pic {legend line=white}; \& \node {$k=3$}; \& \&\\
            \pic {legend line=white}; \& \node {$b = \frac{k\cdot(k-1)}{2}=3$}; \& \&\\
            \pic {legend line=white}; \& \node {$d=k=3$\color{white}dummytextdummyt}; \& \&\\
          };
      }
      \only<3>{
          \small
          \matrix [column sep=2mm, draw, above right, yshift=2cm] at (current bounding box.south east) {
            \pic {legend line=white}; \& \node {$k=3$}; \& \&\\
            \pic {legend line=white}; \& \node {$b = \frac{k\cdot(k-1)}{2}=3$}; \& \&\\
            \pic {legend line=white}; \& \node {$d=k=3$}; \& \&\\
            \pic {legend area=blue!50}; \& \node {$\deg(\ell) = n-r+(k-1) = 4$}; \& \&\\
          };
      }
      \only<4->{
          \small
          \matrix [column sep=2mm, draw, above right, yshift=2cm] at (current bounding box.south east) {
            \pic {legend line=white}; \& \node {$k=3$}; \& \&\\
            \pic {legend line=white}; \& \node {$b = \frac{k\cdot(k-1)}{2}=3$}; \& \&\\
            \pic {legend line=white}; \& \node {$d=k=3$}; \& \&\\
            \pic {legend area=blue!50}; \& \node {$\deg(\ell) = n-r+(k-1) = 4$}; \& \&\\
            \pic {legend area=red!50}; \& \node {$\deg(f_{2_i}) = 1$}; \& \&\\
          };
      }
  \end{tikzpicture}
\end{figure}

\end{frame}
    \end{itemize}
\end{frame}

% FRAME
\begin{frame}{Own Results}{Corollary}
    \begin{center}
        \tcbox[enhanced, skin=enhancedmiddle, arc=0pt, outer arc=0pt, frame hidden, colback=OwlGray!10, boxsep=0mm] {
            \HL is \pNPh with respect to $|L|$
        }
    \end{center}

    \begin{itemize}
        \item The parameterized reduction is also a polynomial reduction
        \item $k$-\textsc{Clique} on regular graphs is \NPh $\xRightarrow{\text{reduction}}$ \HLshort with $|L|=1$ is \NPh
    \end{itemize}
\end{frame}

% FRAME
\begin{frame}{\HLdeg Complexity Picture}
    \begin{figure}[t]
    \centering
    \begin{tikzpicture}[
      node distance=19mm,
      ampersand replacement=\&,
      legend line/.pic={
        \draw[yshift=.5ex,thick,#1] (0, 0) -- (0.6, 0);
      },
      legend area/.pic={
        \draw[yshift=.5ex,fill=#1] (0, -0.2) rectangle (0.6, 0.2);
      }
    ]
      \tikzstyle{every node} = [minimum height=0.7cm,minimum width=1cm,scale=1.0];
      \tikzstyle{result} = [draw, very thick];
      \tikzstyle{corollary} = [draw,dashed];
      \tikzstyle{pNPh} = [fill=red!30];
      \tikzstyle{Wh} = [fill=violet!30];
      \tikzstyle{XP} = [fill=green!30];

      \tikzstyle{edgeXP} = [color=orange,-{Stealth[length=3mm]}]
      \tikzstyle{edgeW} = [color=violet,-{Stealth[length=3mm]}]

      \tikzstyle{dummy} = [color=white, text=white]

      \only<1>{
        % pairs
        \node[corollary,Wh,dummy] (LnB) at (0,0) {\leadnum, \budget};
        \node[corollary,Wh,dummy] (LnD) [right =of LnB] {\leadnum, \safetymarg};
        \node[corollary,Wh,dummy] (BD) [right=of LnD] {\budget, \safetymarg};
        % singletons
        \node[corollary,Wh,dummy] (B) [above left=18mm and 2mm of LnD] {\budget};
        \node[corollary,Wh,dummy] (D) [above right=18mm and 2mm of LnD] {\safetymarg};
        \node[corollary,pNPh,dummy] (Ln) [left=of B] {\leadnum};
        \node[XP, dummy] (Ld) [right=of D] {\leaddeg};
      }
      \only<2>{
        % pairs
        \node[corollary,Wh,label={0:\small{\tt C\,4.7}}] (LnB) at (0,0) {\leadnum, \budget};
        \node[corollary,Wh,label={0:\small{\tt C\,4.7}}] (LnD) [right =of LnB] {\leadnum, \safetymarg};
        \node[corollary,Wh,label={0:\small{\tt C\,4.7}}] (BD) [right=of LnD] {\budget, \safetymarg};
        % singletons
        \node[corollary,Wh,label={0:\small{\tt T\,4.9}}] (B) [above left=18mm and 2mm of LnD] {\budget};
        \node[corollary,Wh,label={0:\small{\tt T\,4.10}}] (D) [above right=18mm and 2mm of LnD] {\safetymarg};
        \node[corollary,Wh,label={0:\small{\tt C\,4.8}}] (Ln) [left=of B] {\leadnum};
        \node[XP, dummy] (Ld) [right=of D] {\leaddeg};
      }
      \only<3->{
        % pairs
        \node[corollary,Wh,label={0:\small{\tt C\,4.7}}] (LnB) at (0,0) {\leadnum, \budget};
        \node[corollary,Wh,label={0:\small{\tt C\,4.7}}] (LnD) [right =of LnB] {\leadnum, \safetymarg};
        \node[corollary,Wh,label={0:\small{\tt C\,4.7}}] (BD) [right=of LnD] {\budget, \safetymarg};
        % singletons
        \node[corollary,Wh,label={0:\small{\tt T\,4.9}}] (B) [above left=18mm and 2mm of LnD] {\budget};
        \node[corollary,Wh,label={0:\small{\tt T\,4.10}}] (D) [above right=18mm and 2mm of LnD] {\safetymarg};
        \node[corollary,pNPh,label={0:\small{\tt C\,4.8}}] (Ln) [left=of B] {\leadnum};
        \node[XP, dummy] (Ld) [right=of D] {\leaddeg};
      }
      % triplets
      \node[result,Wh,label={0:\small{\tt R\,4.6}}] (LnBD) [below=of LnD] {\leadnum, \budget, \safetymarg};
      % edges
      \only<2->{
        \draw[->,edgeW] (LnBD.45) -- (LnB.315);
        \draw[->,edgeW] (LnBD.45) -- (LnD.315);
        \draw[->,edgeW] (LnBD.45) -- (BD.315);
        \draw[->,edgeW] (BD.45) -- (B.315);
        \draw[->,edgeW] (BD.45) -- (D.315);
      }
      \only<2>{
        \draw[->,edgeW] (LnB.45) -- (Ln.315);
      }
      % legend
      \only<1>{
        \small
        \matrix [column sep=2mm, draw, anchor=west, xshift=1.8cm, yshift=0.6cm] at (Ld |- LnBD){
          \pic {legend area=violet!30}; \& \node {\Wh\color{white}ttt}; \& \&\\
        };
      }
      \only<2>{
        \small
        \matrix [column sep=2mm, draw, anchor=west, xshift=1.8cm, yshift=0.6cm] at (Ld |- LnBD){
          \pic {legend area=violet!30}; \& \node {\Wh\color{white}ttt}; \& \&\\
          \pic {legend line=violet}; \& \node {\Wh}; \& \&\\
        };
      }
      \only<3->{
        \small
        \matrix [column sep=2mm, draw, anchor=west, xshift=1.8cm, yshift=0.6cm] at (Ld |- LnBD){
          \pic {legend area=violet!30}; \& \node {\Wh}; \& \&\\
          \pic {legend area=red!30}; \& \node {\pNPh}; \& \&\\
          \pic {legend line=violet}; \& \node {\Wh}; \& \&\\
        };
      }
    \end{tikzpicture}
    \label{fig:complexityPicture}
\end{figure}
\end{frame}

% FRAME
\begin{frame}{Own Results}{Theorem}
    \begin{center}
        \tcbox[enhanced, skin=enhancedmiddle, arc=0pt, outer arc=0pt, frame hidden, colback=OwlGray!10, boxsep=0mm] {
            \HL parameterized by $b$ is in $XP$
        }
    \end{center}

    \begin{itemize}
        \item Try to add every $b$-element subset of edges that can be added between followers
        \begin{itemize}
            \item Check if it is a solution
        \end{itemize}
        \item Can be done in time $n^{\mathcal{O}(b)}$
    \end{itemize}
\end{frame}

% FRAME
\begin{frame}{Own Results}{Theorem}
    \begin{center}
        \tcbox[enhanced, skin=enhancedmiddle, arc=0pt, outer arc=0pt, frame hidden, colback=OwlGray!10, boxsep=0mm] {
            \HL parameterized by $d$ is in $XP$
        }
    \end{center}

    \begin{itemize}
        \item Try every $d$-element subset of followers
        \item Add edges between followers from the subset
        \begin{itemize}
            \item Check if it is a solution
            \item If no, check if a solution can be found by connecting followers outside the subset 
        \end{itemize}
        \item Can be done in time $\mathcal{O}(2^{d^2}) \cdot n^{\mathcal{O}(d)}$
    \end{itemize}
\end{frame}

% FRAME
\begin{frame}{\HLdeg Complexity Picture}
    \begin{figure}[t]
    \centering
    \begin{tikzpicture}[
      node distance=19mm,
      ampersand replacement=\&,
      legend line/.pic={
        \draw[yshift=.5ex,thick,#1] (0, 0) -- (0.6, 0);
      },
      legend area/.pic={
        \draw[yshift=.5ex,fill=#1] (0, -0.2) rectangle (0.6, 0.2);
      }
    ]
      \tikzstyle{every node} = [minimum height=0.7cm,minimum width=1cm,scale=1.0];
      \tikzstyle{result} = [draw, very thick];
      \tikzstyle{corollary} = [draw,dashed];
      \tikzstyle{pNPh} = [fill=red!30];
      \tikzstyle{Wh} = [fill=violet!30];
      \tikzstyle{WhREAL} = [fill=orange!30];
      \tikzstyle{XP} = [fill=green!30];

      \tikzstyle{edgeXP} = [color=orange,-{Stealth[length=3mm]}]
      \tikzstyle{edgeW} = [color=violet,-{Stealth[length=3mm]}]

      \tikzstyle{dummy} = [color=white, text=white]

      % pairs
      \only<1-2>{
        \node[corollary,Wh,label={0:\small{\tt C\,4.7}}] (LnB) at (0,0) {\leadnum, \budget};
        \node[corollary,Wh,label={0:\small{\tt C\,4.7}}] (LnD) [right =of LnB] {\leadnum, \safetymarg};
        \node[corollary,Wh,label={0:\small{\tt C\,4.7}}] (BD) [right=of LnD] {\budget, \safetymarg};
      }
      \only<3->{
        \node[corollary,WhREAL,label={0:\small{\tt C\,4.7}}] (LnB) at (0,0) {\leadnum, \budget};
        \node[corollary,WhREAL,label={0:\small{\tt C\,4.7}}] (LnD) [right =of LnB] {\leadnum, \safetymarg};
        \node[corollary,WhREAL,label={0:\small{\tt C\,4.7}}] (BD) [right=of LnD] {\budget, \safetymarg};
      }
      % singletons
      \only<1>{
        \node[corollary,Wh,label={0:\small{\tt T\,4.9}}] (B) [above left=18mm and 2mm of LnD] {\budget};
        \node[corollary,Wh,label={0:\small{\tt T\,4.10}}] (D) [above right=18mm and 2mm of LnD] {\safetymarg};
      }
      \only<2->{
        \node[result,WhREAL,label={0:\small{\tt T\,4.9}}] (B) [above left=18mm and 2mm of LnD] {\budget};
        \node[result,WhREAL,label={0:\small{\tt T\,4.10}}] (D) [above right=18mm and 2mm of LnD] {\safetymarg};
      }
      \node[corollary,pNPh,label={0:\small{\tt C\,4.8}}] (Ln) [left=of B] {\leadnum};
      \only<1-3>{
        \node[XP, dummy] (Ld) [right=of D] {\leaddeg};
      }
      \only<4->{
        \node[XP,label={0:\small{\tt known}}] (Ld) [right=of D] {\leaddeg};
      }
      % triplets
      \only<1-2>{
        \node[result,Wh,label={0:\small{\tt R\,4.6}}] (LnBD) [below=of LnD] {\leadnum, \budget, \safetymarg};
      }
      \only<3->{
        \node[result,WhREAL,label={0:\small{\tt R\,4.6}}] (LnBD) [below=of LnD] {\leadnum, \budget, \safetymarg};
      }
      % edges
      \draw[->,edgeW] (LnBD.45) -- (LnB.315);
      \draw[->,edgeW] (LnBD.45) -- (LnD.315);
      \draw[->,edgeW] (LnBD.45) -- (BD.315);
      \draw[->,edgeW] (BD.45) -- (B.315);
      \draw[->,edgeW] (BD.45) -- (D.315);
      \only<3->{
        \draw[->,edgeXP] (B.225) -- (LnB.135);
        \draw[->,edgeXP] (D.225) -- (LnD.135);
        \draw[->,edgeXP] (D.225) -- (BD.135);    
        \draw[->,edgeXP] (BD.225) -- (LnBD.135);
      }
      % legend
      \only<1>{
        \small
        \matrix [column sep=2mm, draw, anchor=west, xshift=1.8cm, yshift=0.6cm] at (Ld |- LnBD){
          \pic {legend area=violet!30}; \& \node {\Wh}; \& \&\\
          \pic {legend area=red!30}; \& \node {\pNPh}; \& \&\\
          \pic {legend line=violet}; \& \node {\Wh}; \& \&\\
        };
      }
      \only<2>{
        \small
        \matrix [column sep=2mm, draw, anchor=west, xshift=1.8cm, yshift=0.6cm] at (Ld |- LnBD){
          \pic {legend area=orange!30}; \& \node {\Wh}; \& \&\\
          \pic {legend area=red!30}; \& \node {\pNPh}; \& \&\\
          \pic {legend line=violet}; \& \node {\Wh}; \& \&\\
        };
      }
      \only<3>{
        \small
        \matrix [column sep=2mm, draw, anchor=west, xshift=1.8cm, yshift=0.6cm] at (Ld |- LnBD){
          \pic {legend area=orange!30}; \& \node {\Wh}; \& \&\\
          \pic {legend area=red!30}; \& \node {\pNPh}; \& \&\\
          \pic {legend line=orange}; \& \node {\XP}; \& \&\\
          \pic {legend line=violet}; \& \node {\Wh}; \& \&\\
        };
      }
      \only<4>{
        \small
        \matrix [column sep=2mm, draw, anchor=west, xshift=1.8cm, yshift=0.6cm] at (Ld |- LnBD){
          \pic {legend area=green!30}; \& \node {\XP}; \& \&\\
          \pic {legend area=orange!30}; \& \node {\Wh, \XP}; \& \&\\
          \pic {legend area=red!30}; \& \node {\pNPh}; \& \&\\
          \pic {legend line=orange}; \& \node {\XP}; \& \&\\
          \pic {legend line=violet}; \& \node {\Wh}; \& \&\\
        };
      }
    \end{tikzpicture}
    \label{fig:complexityPicture}
\end{figure}


% \begin{figure}[t]
%   \centering
%   \begin{tikzpicture}[
%     node distance=19mm,
%     ampersand replacement=\&,
%     legend line/.pic={
%       \draw[yshift=.5ex,thick,#1] (0, 0) -- (0.6, 0);
%     },
%     legend area/.pic={
%       \draw[yshift=.5ex,fill=#1] (0, -0.2) rectangle (0.6, 0.2);
%     }
%   ]
%     \tikzstyle{every node} = [minimum height=0.7cm,minimum width=1cm,scale=1.0];
%     \tikzstyle{result} = [draw, very thick];
%     \tikzstyle{corollary} = [draw,dashed];
%     \tikzstyle{pNPh} = [fill=red!30];
%     \tikzstyle{Wh} = [fill=orange!30];
%     \tikzstyle{XP} = [fill=green!30];

%     \tikzstyle{edgeXP} = [color=orange,-{Stealth[length=3mm]}]
%     \tikzstyle{edgeW} = [color=violet,-{Stealth[length=3mm]}]

%     % pairs
%     \node[corollary,Wh,label={0:\small{\tt C\,4.7}}] (LnB) at (0,0) {\leadnum, \budget};
%     \node[corollary,Wh,label={0:\small{\tt C\,4.7}}] (LnD) [right =of LnB] {\leadnum, \safetymarg};
%     \node[corollary,Wh,label={0:\small{\tt C\,4.7}}] (BD) [right=of LnD] {\budget, \safetymarg};
%     % singletons
%     \node[result,Wh,label={0:\small{\tt T\,4.9}}] (B) [above left=18mm and 2mm of LnD] {\budget};
%     \node[result,Wh,label={0:\small{\tt T\,4.10}}] (D) [above right=18mm and 2mm of LnD] {\safetymarg};
%     \node[corollary,pNPh,label={0:\small{\tt C\,4.8}}] (Ln) [left=of B] {\leadnum};
%     \node[XP,label={0:\small{\tt known}}] (Ld) [right=of D] {\leaddeg};
%     % triplets
%     \node[result,Wh,label={0:\small{\tt R\,4.6}}] (LnBD) [below=of LnD] {\leadnum, \budget, \safetymarg};
%     % edges
%     \draw[->,edgeXP] (B.225) -- (LnB.135);
%     \draw[->,edgeXP] (D.225) -- (LnD.135);
%     \draw[->,edgeXP] (D.225) -- (BD.135);    
%     \draw[->,edgeXP] (BD.225) -- (LnBD.135);
    
%     \draw[->,edgeW] (LnBD.45) -- (LnB.315);
%     \draw[->,edgeW] (LnBD.45) -- (LnD.315);
%     \draw[->,edgeW] (LnBD.45) -- (BD.315);
%     \draw[->,edgeW] (BD.45) -- (B.315);
%     \draw[->,edgeW] (BD.45) -- (D.315);
%     % legend
%     \begin{scope}[xshift=12cm,yshift=2cm]
%     \end{scope}
%     \small
%     \matrix [column sep=2mm, draw, anchor=west, xshift=1.8cm, yshift=0.6cm] at (Ld |- LnBD){
%       \pic {legend area=green!30}; \& \node {\XP}; \& \&\\
%       \pic {legend area=orange!30}; \& \node {\Wh, \XP}; \& \&\\
%       \pic {legend area=red!30}; \& \node {\pNPh}; \& \&\\
%       \pic {legend line=orange}; \& \node {\XP}; \& \&\\
%       \pic {legend line=violet}; \& \node {\Wh}; \& \&\\
%     };
%   \end{tikzpicture}
%   \label{fig:complexityPicture}
% \end{figure}
\end{frame}

\section*{Summary}

% FRAME
\begin{frame}{Summary}
  \begin{columns}
    \begin{column}{0.54\textwidth}
      \begin{itemize}
        \item Problem reviewed with respect to the parameterized complexity framework
        \item New complexity and algorithmic results obtained
      \end{itemize}

      % \vskip0pt plus.5fill
      \begin{center}
        Outlook
      \end{center}
      \begin{itemize}
        \item Different centralities or parameters
        \begin{itemize}
          \item vertex cover number
        \end{itemize}
        \item \Wone-hardness for $\lambda$ 
      \end{itemize}
    \end{column}

    \begin{column}{0.44\textwidth}
      \begin{figure}[t]
    \centering
    \begin{tikzpicture}[
      node distance=11.4mm,
      ampersand replacement=\&,
      legend line/.pic={
        \draw[yshift=.5ex,thick,#1] (0, 0) -- (0.6, 0);
      },
      legend area/.pic={
        \draw[yshift=.5ex,fill=#1] (0, -0.2) rectangle (0.6, 0.2);
      }
    ]
      \tikzstyle{every node} = [minimum height=0.49cm,minimum width=0.7cm,scale=0.7];
      \tikzstyle{result} = [draw, thick];
      \tikzstyle{corollary} = [draw,dashed];
      \tikzstyle{pNPh} = [fill=red!30];
      \tikzstyle{Wh} = [fill=orange!30];
      \tikzstyle{XP} = [fill=green!30];

      \tikzstyle{edgeXP} = [color=orange,-{Stealth[length=2mm]}]
      \tikzstyle{edgeW} = [color=violet,-{Stealth[length=2mm]}]

      % pairs
      \node[corollary,Wh,label={0:\small{\tt C\,4.7}}] (LnB) at (0,0) {\leadnum, \budget};
      \node[corollary,Wh,label={0:\small{\tt C\,4.7}}] (LnD) [right =of LnB] {\leadnum, \safetymarg};
      \node[corollary,Wh,label={0:\small{\tt C\,4.7}}] (BD) [right=of LnD] {\budget, \safetymarg};
      % singletons
      \node[result,Wh,label={0:\small{\tt T\,4.9}}] (B) [above left=10.8mm and 2mm of LnD] {\budget};
      \node[result,Wh,label={0:\small{\tt T\,4.10}}] (D) [above right=10.8mm and 2mm of LnD] {\safetymarg};
      \node[corollary,pNPh,label={0:\small{\tt C\,4.8}}] (Ln) [left=of B] {\leadnum};
      \node[XP,label={0:\small{\tt known}}] (Ld) [right=of D] {\leaddeg};
      % triplets
      \node[result,Wh,label={0:\small{\tt R\,4.6}}] (LnBD) [below=of LnD] {\leadnum, \budget, \safetymarg};
      % edges
      \draw[->,edgeXP] (B.225) -- (LnB.135);
      \draw[->,edgeXP] (D.225) -- (LnD.135);
      \draw[->,edgeXP] (D.225) -- (BD.135);    
      \draw[->,edgeXP] (BD.225) -- (LnBD.135);
      
      \draw[->,edgeW] (LnBD.45) -- (LnB.315);
      \draw[->,edgeW] (LnBD.45) -- (LnD.315);
      \draw[->,edgeW] (LnBD.45) -- (BD.315);
      \draw[->,edgeW] (BD.45) -- (B.315);
      \draw[->,edgeW] (BD.45) -- (D.315);
      % legend
      % \begin{scope}[xshift=12cm,yshift=2cm]
      % \end{scope}
      % \small
      % \matrix [column sep=2mm, draw, anchor=west, xshift=0.8cm, yshift=-4.8cm] at (0,0){
      %   \pic {legend area=green!30}; \& \node {\XP}; \& \&\\
      %   \pic {legend area=orange!30}; \& \node {\Wh, \XP}; \& \&\\
      %   \pic {legend area=red!30}; \& \node {\pNPh}; \& \&\\
      %   \pic {legend line=orange}; \& \node {\XP}; \& \&\\
      %   \pic {legend line=violet}; \& \node {\Wh}; \& \&\\
      % };
    \end{tikzpicture}
    \label{fig:complexityPicture}
\end{figure}
    \end{column}
  \end{columns}

  \uncover<2->{
    \begin{center}
      \textbf{Thank you for your attention!}
    \end{center}
  }
\end{frame}

\appendix
% FRAME
\begin{frame}{Questions from the opponent}
\begin{alertblock}{Question 1}
    \begin{quote}
    Dotaz 1
    \end{quote}
\end{alertblock}

\begin{tcolorbox}%[colback=green!5,colframe=green!40!black,title=A nice heading]
    Rozepsaná Odpověď
\end{tcolorbox}

\begin{alertblock}{Question 2}
    \begin{quote}
    Dotaz 2
    \end{quote}
\end{alertblock}

\begin{tcolorbox}%[colback=green!5,colframe=green!40!black,title=A nice heading]
    Rozepsaná Odpověď
\end{tcolorbox}
\end{frame}




% TMP samples

% ----------------------------------------------------
% \subsection
% [Short First Subsection Name]
% {Subsection name}

% % FRAME
% \begin{frame}{The \HL problem}%{Subtitles are optional}
%   % - A title should summarize the slide in an understandable fashion
%   %   for anyone how does not follow everything on the slide itself.

%   \begin{itemize}
%   \item
%     Use \texttt{itemize} a lot.
%   \item
%     Use very short sentences or short phrases.
%   \end{itemize}
% \end{frame}


% % FRAME
% \begin{frame}{Make Titles Informative.}{Subtitle}
%   You can create overlays\dots
%   \begin{itemize}
%   \item using the \texttt{pause} command:
%     \begin{itemize}
%     \item
%       First item.
%       \pause
%     \item    
%       Second item.
%     \end{itemize}
%   \item
%     using overlay specifications:
%     \begin{itemize}
%     \item<3->
%       First item.
%     \item<4->
%       Second item.
%     \end{itemize}
%   \item
%     using the general \texttt{uncover} command:
%     \begin{itemize}
%       \uncover<5->{\item
%         First item.}
%       \uncover<6->{\item
%         Second item.}
%     \end{itemize}
%   \end{itemize}
% \end{frame}


% % FRAME
% \begin{frame}{<title>}
%   \begin{columns}
%     \begin{column}{0.49\textwidth}
%       \begin{alertblock}{title}
%         tu
%       \end{alertblock}
%     \end{column}

%     \begin{column}{0.49\textwidth}
%       \begin{exampleblock}{title}
%         tam
%       \end{exampleblock}
%     \end{column}
%   \end{columns}
% \end{frame}


% % FRAME
% \begin{frame}[standout]
%   Thank you for your attention!
% \end{frame}

\end{document}
