% FRAME
\begin{frame}{Questions from the opponent}
\begin{alertblock}{Question}
    \begin{quote}
        V definici 2.11 definujete centralitu měřenou stupněm uzlu jako hodnotu danou stupněm daného vrcholu (1).
        Články, na které se odkazujete v konstrukcích a důkazech ovšem tuto míru definují jinak (2) (zdroje [2, 51]).
        Není tato odlišná definice problémem v důkazech, které v práci uvádíte?
        Prodiskutujte prosím odlišnosti a navrhněte řešení.
    \end{quote}
\end{alertblock}

\centering
(1) $c(G,v) = \deg v$

\centering
(2) $c(G,v) = \deg \frac{v}{|V|-1}$
\end{frame}


% FRAME
\begin{frame}{Questions from the opponent}
    \centering
    (1) $c(G,v) = \deg v$
    \medspace\medspace\medspace\medspace\medspace\medspace\medspace\medspace\medspace
    (2) $c(G,v) = \deg \frac{v}{|V|-1}$

    \begin{exampleblock}{Answer}
        Odlišná definice není problémem v důkazech, které v práci uvádím.

        Normalizační faktor $\frac{1}{|V|-1}$ ve vztahu (2) představuje pro daný graf konstantu
        a tedy nemění relativní pořadí vrcholů oproti pořadí určeného pomocí vztahu (1).
        Po přidání/odebrání vrcholů do/z grafu se sice tento faktor změní, zůstane však stejný pro všechny vrcholy v grafu
        a relativní pořadí oproti vzathu (1) se tak nezmění.
        Během zkoumání problému \HL jde právě o relativní pořadí vrcholů mezi sebou,
        čili vztah (1) nepředstavuje v tomto ohledu vůči vztahu (2) žádný rozdíl,
        avšak umožňuje jednodušší argumentaci díky možné záměně centrality vrcholu a stupně vrcholu.

        Vztah (1) je také použitý v definici ve zdroji [3].


    \end{exampleblock}

    % \begin{tcolorbox}%[colback=green!5,colframe=green!40!black,title=A nice heading]
    %     Rozepsaná Odpověď
    % \end{tcolorbox}
\end{frame}