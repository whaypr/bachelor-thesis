%% This is the ctufit-thesis example file. It is used to produce theses
%% for submission to Czech Technical University, Faculty of Information Technology.
%%
%% Get the newest version from
%% https://gitlab.fit.cvut.cz/theses-templates/FITthesis-LaTeX
%%
%%
%% Copyright 2021, Eliska Sestakova and Ondrej Guth
%%
%% This work may be distributed and/or modified under the
%% conditions of the LaTeX Project Public Licenese, either version 1.3
%% of this license or (at your option) any later version.
%% The latest version of this license is in
%%  https://www.latex-project.org/lppl.txt
%% and version 1.3 or later is part of all distributions of LaTeX
%% version 2005/12/01 or later.
%%
%% This work has the LPPL maintenance status `maintained'.
%%
%% The current maintainer of this work is Ondrej Guth.
%% Contact ondrej.guth@fit.cvut.cz for bug reports.
%% Alternatively, submit bug reports into the tracker at
%% https://gitlab.fit.cvut.cz/theses-templates/FITthesis-LaTeX/issues
%%
%%

%%%%%%%%%%%%%%%%%%%%%%%%%%%%%%%%%%%%%%%%%
% CLASS OPTIONS
% language: czech/english/slovak
% thesis type: bachelor/master/dissertation
% colour: bw for black&white OR no option for default colour scheme
%%%%%%%%%%%%%%%%%%%%%%%%%%%%%%%%%%%%%%%%%
\documentclass[english,bachelor,unicode]{ctufit-thesis}


%%%%%%%%%%%%%%%%%%%%%%%%%%%%%%%%%%
% FILL IN THIS INFORMATION
%%%%%%%%%%%%%%%%%%%%%%%%%%%%%%%%%%
\ctufittitle{Hiding Leaders in Covert Networks: A Computational Complexity Perspective} % replace with the title of your thesis
\ctufitauthorfull{Patrik Drbal} % replace with your full name (first name(s) and then family name(s) / surname(s)) including academic degrees
\ctufitauthorsurnames{Drbal} % replace with your surname(s) / family name(s)
\ctufitauthorgivennames{Patrik} % replace with your first name(s) / given name(s)
\ctufitsupervisor{Ing.\,Šimon Schierreich} % replace with name of your supervisor/advisor (include academic degrees)
\ctufitdepartment{Department of Theoretical Computer Science} % replace with the department of your defence
\ctufityear{2023} % replace with the year of your defence
\ctufitdeclarationplace{Prague} % replace with the place where you sign the declaration
\ctufitdeclarationdate{\today} % replace with the date of signature of the declaration
\ctufitabstractENG{
    In this thesis, we provide an introduction to the topic of covert networks and their analysis,
    with emphasis on the \HL problem
    which we study with respect to the degree centrality measure
    and using the framework of parameterized computational complexity.

    We analyze results for the \HL problem provided in the literature and put them into a perspective of the parameterized complexity.
    We also point out that there are multiple definitions of the problem
    and review some of the results from the literature with respect to the definition we have picked.

    We then bring new results to the domain by showing that
    the problem is \Wh when parameterized by $b+d$ even when $|L|=1$, and \pNPh with respect to $|L|$,
    where $b$ denotes a number of edges allowed to be added into given network,
    $d$ denotes a number of followers needed to have centrality at least as high as any of leaders
    and $L$ denotes a set of leaders.
    We also show that the problem is in \XP when parameterized by $b$ or $d$.

    In the end, we put together our results and their corollaries -- for parameters $|L|$, $b$, $d$ --
    and results from the literature -- for parameter $\lambda$, where $\lambda$ denotes a maximum degree among leaders --
    and visualize them altogether in an overviewing graph,
    providing an easily accessible summary of the current state of the research of \HL
    given the decision version of the problem, the degree centrality and the four aforementioned parameters.
}
\ctufitabstractCZE{
    V této práci poskytujeme úvod do problematiky skrytých sítí (\emph{covert networks}) a jejich analýzy,
    s důrazem na problém \HL, který zkoumáme s ohledem na míru centrality měřenou stupněm uzlu (\emph{degree centrality})
    a pomocí frameworku parametrizované výpočetní složitosti.

    Analyzujeme výsledky pro problém \HL uvedené v literatuře a dáváme je do perspektivy parametrizované složitosti.
    Zdůrazňujeme také, že existuje více definic tohoto problému a přezkoumáváme některé výsledky z literatury
    s ohledem na námi vybranou definici.

    Dále přinášíme nové výsledky a ukazujeme, že problém je \textsf{W}[1]-těžký pro parametr $b+d$ i když $|L|=1$,
    a \textsf{para-NP}-těžký vzhledem k $|L|$,
    kde $b$ označuje počet hran, které mohou být přidány do dané sítě,
    $d$ označuje požadovaný počet následovníků, kteří musí mít centralitu alespoň tak vysokou jako kterýkoli z vůdců,
    a $L$ označuje množinu vůdců.
    Také ukazujeme, že problém je v \XP při parametrizaci parametry $b$ nebo $d$.

    V závěru práce spojujeme naše výsledky a jejich důsledky -- pro parametry $|L|$, $b$, $d$ --
    s výsledky z literatury -- pro parametr $\lambda$, kde $\lambda$ označuje maximální stupeň mezi vůdci --
    a vizualizujeme je společně v přehledném grafu,
    který poskytuje snadno dostupné shrnutí současného stavu výzkumu \HL problému
    vzhledem k rozhodovací verzi tohoto problému, centralitě měřené stupněm uzlu a čtyřem výše uvedeným parametrům.
}
\ctufitkeywordsENG{Hiding Leaders problem, covert networks, parameterized complexity, degree centrality}
\ctufitkeywordsCZE{problém Hiding Leaders, covert networks, parametrizovaná složitost, degree centrality}
%%%%%%%%%%%%%%%%%%%%%%%%%%%%%%%%%%
% END FILL IN
%%%%%%%%%%%%%%%%%%%%%%%%%%%%%%%%%%


%%%%%%%%%%%%%%%%%%%%%%%%%%%%%%%%%%
% CUSTOMIZATION of this template
% Skip this part or alter it if you know what you are doing.
%%%%%%%%%%%%%%%%%%%%%%%%%%%%%%%%%%
\RequirePackage{iftex}[2020/03/06]
\iftutex % XeLaTeX and LuaLaTeX
    \RequirePackage{ellipsis}[2020/05/22] %ellipsis workaround for XeLaTeX
\else
    \RequirePackage[utf8]{inputenc}[2018/08/11] %this file encoding
    \RequirePackage{lmodern}[2009/10/30] % vector flavor of Computer Modern font
\fi

% hyperlinks
\RequirePackage[pdfpagelayout=TwoPageRight,colorlinks=false,allcolors=decoration,pdfborder={0 0 0.1}]{hyperref}[2020-05-15]

% uncomment the following to hide all hyperlinks
% \RequirePackage[pdfpagelayout=TwoPageRight,hidelinks]{hyperref}[2020-05-15]

\RequirePackage{pdfpages}[2020/01/28]

\setcounter{secnumdepth}{4} % numbering sections; 4: subsubsection
%%%%%%%%%%%%%%%%%%%%%%%%%%%%%%%%%%
% CUSTOMIZATION of this template END
%%%%%%%%%%%%%%%%%%%%%%%%%%%%%%%%%%


%%%%%%%%%%%%%%%%%%%%%%
% DEMO CONTENTS SETTINGS
% You may choose to modify this part.
%%%%%%%%%%%%%%%%%%%%%%
\usepackage{dirtree}
\usepackage{lipsum,tikz}
    \usetikzlibrary{shapes.geometric}
    \usetikzlibrary{calc}
    \usetikzlibrary{arrows.meta,bending,positioning}
\usepackage{csquotes}
\usepackage[style=iso-numeric,backend=bibtex]{biblatex}
\addbibresource{text/bib-database.bib}
\usepackage{listings} % typesetting of sources
% \usepackage{minted} % typesetting of sources
\usepackage{todonotes}
\usepackage{xspace}
\usepackage{subcaption}
\usepackage{xevlna}
\usepackage[most]{tcolorbox}

%theorems, definitions, etc.
\theoremstyle{plain}
\newtheorem{theorem}{Theorem}
\newtheorem{lemma}[theorem]{Lemma}
\newtheorem{corollary}[theorem]{Corollary}
\newtheorem{proposition}[theorem]{Proposition}
\newtheorem{definition}[theorem]{Definition}
\theoremstyle{definition}
\newtheorem{example}[theorem]{Example}
\theoremstyle{remark}
\newtheorem{note}[theorem]{Note}
\newtheorem*{note*}{Note}
\newtheorem{remark}[theorem]{Remark}
\newtheorem*{remark*}{Remark}
\numberwithin{theorem}{chapter}
%theorems, definitions, etc. END
%%%%%%%%%%%%%%%%%%%%%%
% DEMO CONTENTS SETTINGS END
%%%%%%%%%%%%%%%%%%%%%%


\begin{document} 
\frontmatter\frontmatterinit % do not remove these two commands

\includepdf[pages={1-}]{assignment-include.pdf} % replace that file with your thesis assignment provided by study office

\thispagestyle{empty}\cleardoublepage\maketitle % do not remove these three commands

\imprintpage % do not remove this command

\tableofcontents % do not remove this command


%%%%%%%%%%%%%%%%%%%%%%
% list of other contents: figures, tables, code listings, algorithms, etc.
% add/remove commands accordingly
%%%%%%%%%%%%%%%%%%%%%%
\listoffigures % list of figures
\begingroup
\let\clearpage\relax
% \listoftables % list of tables
% \lstlistoflistings % list of source code listings generated by the listings package
% \listoflistings % list of source code listings generated by the minted package
\endgroup
%%%%%%%%%%%%%%%%%%%%%%
% list of other contents END
%%%%%%%%%%%%%%%%%%%%%%


%%%%%%%%%%%%%%%%%%%
% ACKNOWLEDGMENT
% FILL IN / MODIFY
% This is a place to thank people for helping you. It is common to thank your supervisor.
%%%%%%%%%%%%%%%%%%%
\begin{acknowledgmentpage}
    On this place, I would like to express my sincere gratitude to the supervisor of this work, Ing. Šimon Schierreich,
    for generously providing me with his knowledge, expertise and time.
    Especially for giving me consultations whenever I needed, as
    they always strengthen my motivatation to keep working and I always left them a little smarter,
    for willingly and helpfully pointing out mistakes I made, and
    for being my guide to the whole process, letting me dive deeper into the fascinating world of
    theoretical computer science.

    I further wish to thank Ing. Dana Vynikarová, Ph.D. for teaching me all the formal aspects of writing
    and structuring scientific text, and for her an ever-present smile on her face
    and willingness to help.

    I am also grateful to my parents for always believing in me and supporting me.

    Last but not least, my deepest thanks belong to my girlfriend Dorotka for all the love and support she gives me.
    Without her, I would hardly be able to achieve such a peace of mind and determination that
    allowed me to properly work on this thesis.
\end{acknowledgmentpage} 
%%%%%%%%%%%%%%%%%%%
% ACKNOWLEDGMENT END
%%%%%%%%%%%%%%%%%%%


%%%%%%%%%%%%%%%%%%%
% DECLARATION
% FILL IN / MODIFY
%%%%%%%%%%%%%%%%%%%
% INSTRUCTIONS
% ENG: choose one of approved texts of the declaration. DO NOT CREATE YOUR OWN. Find the approved texts at https://courses.fit.cvut.cz/SFE/download/index.html#_documents (document Declaration for FT in English)
% CZE/SLO: Vyberte jedno z fakultou schvalenych prohlaseni. NEVKLADEJTE VLASTNI TEXT. Schvalena prohlaseni najdete zde: https://courses.fit.cvut.cz/SZZ/dokumenty/index.html#_dokumenty (prohlášení do ZP)
\begin{declarationpage}
    I hereby declare that the presented thesis is my own work and that I have cited all sources of information in accordance with the Guideline
    for adhering to ethical principles when elaborating an academic final thesis.
    I acknowledge that my thesis is subject to the rights and obligations stipulated by the Act No. 121/2000 Coll.,
    the Copyright Act, as amended, in particular that the Czech Technical University in Prague has the right to conclude
    a license agreement on the utilization of this thesis as a school work under the provisions of Article 60 (1) of the Act.
\end{declarationpage}
%%%%%%%%%%%%%%%%%%%
% DECLARATION END
%%%%%%%%%%%%%%%%%%%


\printabstractpage % do not remove this command


%%%%%%%%%%%%%%%%%%%
% ABBREVIATIONS
% FILL IN / MODIFY
% OR REMOVE ENTIRELY
% List the abbreviations in lexicography order.
%%%%%%%%%%%%%%%%%%%
\chapter{List of abbreviations}

\begin{tabular}{rl}
    CNA  & Covert network analysis\\
    FPT  & Fixed-parameter tractable\\
    HL   & Hiding Leaders\\
    SNA  & Social network analysis
\end{tabular}
%%%%%%%%%%%%%%%%%%%
% ABBREVIATIONS END
%%%%%%%%%%%%%%%%%%%


\mainmatter\mainmatterinit % do not remove these two commands


%%%%%%%%%%%%%%%%%%%
% THE THESIS
% MODIFY ANYTHING BELOW THIS LINE
%%%%%%%%%%%%%%%%%%%
%---------------------------------------------------------------
\chapter*{Introduction}\addcontentsline{toc}{chapter}{Introduction}\markboth{Introduction}{Introduction}
%---------------------------------------------------------------

\setcounter{page}{1}

Covert networks, or covert organizations, are social structures whose one of the main concerns is to operate in secret,
concealed from view of public, view of governments and intelligence agencies or view of any other unwanted subject.
Covert network analysis (CNA), as an important part of social network analysis (SNA),
is then a set of tools and techniques used to study covert networks and their members,
aiming to uncovering connections between the members and identifying those with a great
influence and thus importance, despite many connections remain unknown during this identification.
Such influential users are called leaders.
One of the most used techniques in the SNA toolset, and CNA in particular, for detecting leaders of covert networks
are \emph{centrality measures},
the concept introduced by Bavelas \cite{Bavelas1948} describing how important, or central,
a member of given network inside this network is.
Being crucial for their working, leaders try to stay undetected by such measures,
for which they use various counter-techniques such as
network modification by adding or deleting edges, or creating whole new networks from scratch.

In this thesis, we survey covert networks and hiding inside of them from a perspective of the \HL problem (\HLshort), first defined by
Waniek et al. \cite{Waniek2017} and further studied by various authors \cite{Dey2019,Waniek2021,Mohan2023}.
More concretely, our main concern is the \HL problem with respect to a degree centrality measure
and using the framework of parameterized computational complexity.
Because the problem is \NPh by itself, we expose it to various parameters and
see how these parameters affect its computational complexity.
Another goal of this thesis is to analyze the problem variants studied in the literature
and describe the current state of the research in the language of the parameterized complexity framework.
Profound description of the \HL problem can be found in Chapter \ref{chapter:ProblemStatement}, for introduction to
the theory of parameterized complexity, please refer to Section \ref{section:ParamComp}.

%---------------------------------------------------------------
\chapter{Literature review}
%---------------------------------------------------------------

In this chapter, we bring a thorough examination of the literature on two topics;
social network analysis together with covert networks, and parameterized complexity.
The study of the former topic provides an useful context for the \HL problem, whereas
the later topic gives us tools to better reason about \NPh problems, among which, as we will see, the \HL problem surely belongs.

%---------------------------------------------------------------
\section{Social network analysis \& covert networks}

First of all, we mention the publication from Morselli \cite{Morselli2009} that the reader can refer to get
an extensive study of criminal networks.
Next, for recent and exhaustive overview of SNA techniques used for covert network disruption,
the reader can use the paper from Ficara et al. \cite{Ficara2022}.

Arguably the most extensive line of research is that in which authors use covert networks to analyze structures consisting of
harmful actors.
That can be terorist organizations or other structures interested in involving into illegal activities. These networks are
also called \emph{dark networks} \cite{Raab2003}.
Indeed, research in this direction have positive impact on understanding such harmful organisations,
shedding light on their inner processes and the most potentially hazardous actors and helping in dismantling the entire network,
preventing further operations and minimizing damage done.
The understanding of how a covert network works inside can lead to understanding of how it can be broken.
Among many others, we mention work of these authors \cite{Waniek2017,Dey2019,Raab2003,Lindelauf2009,Xu2005,Ressler2006, SaavedraNieves2023}.
However, as noted by Saavedra-Nieves and Casas-Méndez \cite{SaavedraNieves2023}, this view on covert networks as organizations with harmful
intends is not the only possible since covert populations can consist for example of drug users, illegal immigrants, persecuted jews,
people with infectious diseases, activists, or ravers \cite{Oliver2014}.
The usefulness of covert network analysis then lingers in analyzing even such populations.
CNA tools, together with other methods of SNA, can be then used even, for example, for analyzing sports teams \cite{Buldú2019}.

The centrality measurement techniques mentioned in the thesis introduction are widely used because
finding the most important vertices within a given network is a natural approach when
studying real-world social networks \cite{Crescenzi2016}.
The most classical centrality measures are degree \cite{Shaw1954}, closeness \cite{Beauchamp1965},
betweenness \cite{Anthonisse1971,Freeman1977}, and core \cite{Seidman1983} centrality.
On top of the standard centrality measures, there are more recent and advanced techniques suitable for measuring one's rank within network,
for example the \emph{Game of Thieves} algorithm from Mocanu et al. \cite{Mocanu2018}, which can be used even within massive networks.
For the sake of completeness, we also mention that there also exist centrality measures on edges, e.g.,
\emph{WERW-Kpath} algorithm from {De Meo} \cite{DeMeo2013}.
The two aforementioned algorithms are more reviewed by Ficara et al. \cite{Ficara2021}.

After detecting influential members within given network using centrality measures,
it is often desirable to talk about their influence in this networks, for which
various \emph{models of influence} can be used.
Among frequently used models belong the \emph{Independent Cascade} model and \emph{Linear Threshold} model.
More on this topic can be found in the book from Easley and Kleinberg \cite{Easley2010}.

Apart from the centrality measures, there are other important tools SNA has to offer.
One of them are \emph{node similarity measures} which are with advantage used in \emph{link prediction} \cite{Zhou2009,Wang2014}.
Link prediction also plays an important part of research of covert networks
as it allows to predict the existence of otherwise hidden connections between members, potentially leading to
identification of important members and relations between them.
This is why covert network users might want to manipulate the network to minimize efficiency of such techniques \cite{Zhou2019}.
Besides CNA, link prediction have other useful applications for example in recommendation systems \cite{Huang2005,Talasu2017}.

Next, we mention that leaders of covert networks often (but not always \cite{Fatih2012}) face what is called
the \emph{efficiency/security trade-off} \cite{Morselli2007} --
the dilemma between staying sufficiently hidden, while keeping enough influence over the network --
which is the often-mentioned topic in the literature about covert networks \cite{Waniek2017,Lindelauf2009,Crossley2012}.
However, the ways in which this trade-off is addressed vary in the literature.
Waniek et al. \cite{Waniek2017}, for example, approach this by modeling the security from the perspective of
centrality measures (centrality measure based secrecy), whereas the efficiency from the perspective of models of influence.
They then show how to construct a network designed specifically to hide its leaders, while
keeping their ability to influence the rest of the network.
Lindelauf et al. \cite{Lindelauf2009}, on the other hand, combine graph and game theory to model this dilemma.

When studying covert networks, we often talk about two parties, the \emph{evaders} and the \emph{seekers};
evaders are members (often leaders) of some covert network;
seekers are then those who try to identify the evaders.
Evaders are typically aware of the importance of their network structure, as can be seen from
the efficiency/security dilemma.
However, when talking about seekers and evaders, authors typically consider these three scenarios:
(i) Evaders act unstrategically in the sense that they are unaware of
the existence of seekers and SNA tools seekers use to reveal them;
(ii) Evaders behave as strategic actors, well-awared of the centrality analysis
seekers use against them \cite{Waniek2017,Dey2019,Waniek2016,Dey2020}.
(iii) Both seekers and evaders are strategic.
To this end, we mention the work from Waniek et al. \cite{Waniek2021} in which the authors propose a strategy
which tells the seekers which centrality measures they should use
to maximize the chances of detecting a leader of a covert network.

In the terminology of this section, the \HL problem describes a covert network consisting of strategic leaders
in the role of evaders, which are trying to evade being detected by the use of centrality measuring techniques,
for which they modify their network by adding new connections between network members.
The leaders do not delete any edges because they face the efficiency/security trade-off and
deleting edges may decrease their efficiency in the network.
The problem is not bound with any influence model and hence can be study independently of it,
leaving the ``efficiency'' part of the efficiency/security trade-off untouched.


%---------------------------------------------------------------
\section{Parameterized computational complexity}

What follows is a sumamry of the literature on the topic of parameterized complexity.
For an introduction to the topic itself, please refer to Section \ref{section:ParamComp}.

The foundations of parameterized complexity were laid by Downey and Fellows
in the series of papers from years 1992 to 1995 \cite{Downey1992,Downey1995.1,Downey1995.2,Downey1993,Downey1995.4},
which the authors further presented later in 1999 in their book \cite{Downey1999},
which was refined and once more published in 2013 \cite{Downey2013}.
Other relevant and potentially useful literature on this topic are
two books by Flum with Grohe \cite{Flum2006} and by Niedermeier \cite{Niedermeier2006} from 2006, book by Hans et al. \cite{Hans2012} from 2012,
book by Cygan et al. \cite{Cygan2015} from 2015 (which to a great extent covers knowledge from the previous literature)
and two books by Haan \cite{Haan2019} and by Fomin et al. (focused mainly on kernelization) \cite{Fomin2019} from 2019.
However, keep in mind that there is a highly active line of research in the field of parameterized complexity, so,
as researchers keep making new discoveries, information presented in the work of mentioned authors may not necessarily be up-to-date.

%---------------------------------------------------------------
\chapter{Preliminaries}
%---------------------------------------------------------------

Before proceeding to the main parts of the work,
let us begin with a couple of definitions and theory the reader might find useful to fully understand later chapters.
This chapter introduces preliminaries on graph theory, including graph problems and centrality measures, and 
classical computational complexity followed by parameterized complexity.


%---------------------------------------------------------------
\section{Graph theory}

This section introduces basic concepts from graph theory.
For a thorough revision on this topic, please refer to the monograph of Diestel \cite{Diestel2018}.

% Graph
\begin{definition}[Simple graph]
    A \emph{simple graph} is an ordered pair $G=(V,E)$, where $V$ and $E$ are disjoint sets,
    $V$ is a finite nonempty set of arbitrary items called vertices (or nodes) and 
    $E$ is a finite set of unordered pairs of vertices called edges, $E \subseteq \{\{x,y\} \mid x,y \in V \wedge x \neq y\}$.
\end{definition}

Each graph used in this thesis is simple graph, for this reason,
we omit the word ``simple'' through the text and call each graph just ``graph''.
Also, the terms ``graph'' and ``network'' are interchangeable in this thesis as we only use graphs to describe networks.
By network we mean such graph, where vertices represents members of the network 
and edges represents, in some sense, a connection between two given members, e.g., enabling their communication.

Sets of all vertices $V$ and edges $E$ of graph $G$ are typically denoted as $V(G)$ and $E(G)$ respectively.
A number of vertices in graphs $G$ is typically denoted as $n$, $|V(G)| = n$ and
a number of edges in $G$ is typically denoted as $m$, $|E(G)| = m$.

% Degree
\begin{definition}[Vertex degree]
    A degree of vertex $v$ in some graph $G$, denoted as $\deg(v)$, is the number of neighbors of $v$.
    In other words, it is the number of edges vertex $v$ is a part of, i.e.,
    $\Big|\{x \in V(G) \vert \{v,x\} \in E(G) \vee \{x,v\} \in E(G)\}\Big|$.
\end{definition}

A set of neighbors of some vertex $v$ is often denoted as $N(v)$, so then $\deg(v) = |N(v)|$.

% r-regular graph
\begin{definition}[$r$-regular graph]
    A graph $G$ is \emph{$r$-regular} if the degree of each vertex from $V(G)$ is $r$, i.e., $\forall v \in V(G) \colon \deg(v) = r$.
\end{definition}

% Regular graph
\begin{definition}[Regular graph]
    A graph $G$ is \emph{regular} if there exists some $r \in \mathbb{N}$ for which~$G$ is $r$-regular.
\end{definition}

% Complement graph
\begin{definition}[Complement graph]
    A \emph{complement graph} $\overline{G}$ of graph $G$ is the graph on vertices $V(G)$,
    where two vertices are adjacent (there is an edge between them) if and only if
    they are not adjacent in $G$, i.e., $\forall e \in V(G) \times V(G) \colon e \in E(\overline{G}) \Leftrightarrow e \notin E(G)$.
\end{definition}

% Induced subgraph
\begin{definition}[Induced subgraph]
    An \emph{induced subgraph} $G[S]$ of graph $G$ is the graph on vertices $V(G[S]) = S \subset V(G)$,
    where two vertices are adjacent if and only if they are adjacent in $G$, i.e.,
    $(\forall u,v \in S)(\{u,v\} \in E(G[S]) \Leftrightarrow \{u,v\} \in E(G))$.
\end{definition}

% Complete graph
\begin{definition}[Complete graph]
    A \emph{complete graph} is the graph $G$ if there is an edge between every pair of vertices from $V(G)$, i.e.,
    $(\forall u,v \in V(G))(\{u,v\} \in E(G))$.
\end{definition}

% Clique
\begin{definition}[Clique]
    A clique $C$ is the subgraph of graph $G$ where $G[E(C)]$ is a complete graph.
\end{definition}

% Vertex cover
\begin{definition}[Vertex cover]
    A \emph{vertex cover} of graph $G$ is the subset $S \subseteq V(G)$ where each edge from $E(G)$ is
    covered by some vertex from $S$. In other words, at least one endpoint of each edge is present in $S$, i.e.,
    $(\forall \{u,v\} \in E(G))(u \in S \vee v \in S)$.
\end{definition}


\subsection{Graph problems}

What follows is a definition of the \NPh graph problem we will use in Proof \ref{proofDB}.

% k-Clique
\begin{definition}[$k$-\textsc{Clique} problem]
    Given graph $G$, the $k$-\textsc{Clique} problem is to determine if
    there exists a clique $C$ in $G$ where $|V(C)| = k$.
\end{definition}

There are many other graph problems related to cliques. The $k$-\textsc{Clique} problem is a decision problem but
there are also optimization variants, for example the \textsc{Maximum Clique} problem.


\subsection{Centrality measures}

Centrality measure, or centrality for short, is a measure from graph theory which is widely used in social network analysis.
Generally, centrality is a function $c: G \times V \rightarrow \mathbb{R}$ which describes the importance of a node in given network.
With centrality, we can measure a ranking (position among other nodes) of given node within their network.
What follows is a definition of a degree centrality introduced by Shaw \cite{Shaw1954}, but there are other centralities used in SNA like
closeness~\cite{Beauchamp1965}, betweenness \cite{Anthonisse1971,Freeman1977} or core \cite{Seidman1983} centrality.

\begin{definition}[Degree centrality]
    A degree centrality measures an importance of a vertex by its degree.
    A degree centrality of the vertex $v$ in network $G$ is defined as:
    $$c_{deg}(G, v) = \deg(v).$$
\end{definition}


%---------------------------------------------------------------
\section{Classical computational complexity}

In this section, we briefly describe the two most fundamental classical complexity classes, \Po and \NP.
Detailed revision on this topic can be found in the textbook from Arora and Barak \cite{Arora2009}.

Let us start with defining the \textsf{DTIME} complexity class, which we then use to define the \Po class.

\begin{definition}[\textsf{DTIME}]
    A complexity class \textsf{DTIME}(f(n)) is the set of all decision problems that are computable in time
    $c \cdot f(n)$ for some constant $c > 0$ and some function $f \colon \mathbb{N} \rightarrow \mathbb{N}$.
\end{definition}

% P
\begin{definition}[\Po]
    A complexity class \Po is the set of all decision problems that are in class \textsf{DTIME}($n^c$)
    for any $c \geq 1$, i.e.
    $\cup_{c \geq 1} \textsf{DTIME}(n^c)$.
\end{definition}

Informally, complexity class \Po consists of all decision problems that can be solved in polynomial time.
Problems for which exists in practice efficient enough algorithm are called \emph{tractable}.

% NP
\begin{definition}[\NP]
    A complexity class \NP is the set of all decision problems $L$ for which there exist
    a polynomial $p \colon \mathbb{N} \rightarrow \mathbb{N}$ and a deterministic Turing machine $M$
    such that for every instance $x$ of $L$ $x$ is a $yes$-instance if and only if there exists
    a polynomial-size solution $u$ of $x$ that can be verified on $M$ in polynomial time.
    Such solution $u$ is then called the \emph{certificate} or the \emph{witness}.
\end{definition}

Informally, complexity class \NP consists of all decision problems that can be solved in polynomial time
when multiple steps can be done in parallel.
Note that $\Po \subseteq \NP$, because, for any problem from \Po, it takes only a polynomial amount of time to solve
it in the first place.

At the end of this section, let us briefly recall computational complexity terms \emph{hardness} and \emph{completeness}.
A problem $p$ is, given some complexity class \textsf{C}, \textsf{C-hard} if for any problem $c$ from~\textsf{C}
there exists a polynomial reduction from $c$ to $p$.
A problem $p$ is then \textsf{C-complete} if it is \textsf{C-hard} and it belongs to \textsf{C} at the same time.


%---------------------------------------------------------------
\section{Parameterized computational complexity}\label{section:ParamComp}

We use the textbook by Cygan et al. \cite{Cygan2015} as our main source of information on this topic since it
provides arguably the most complex and comprehensive overview of the theory of parameterized complexity and,
in many cases, presents the state of the art in the field.

The classical complexity analysis is often insufficient when dealing with \NPh problems.
The main reason is that there are \NPh problems that are in some sense harder than the others.
This inability of the classical approach to distinguish between the hardness of various \NPh problems
led to a development of the parameterized complexity theory.
Parameterized complexity is a direct generalization of the classical complexity theory and it arms its users
with the ability to analyze running time of algorithms, and thus computational complexities of problems
they try to solve, in a finer detail.
It gives us tools which help us to distinguish between different difficulty levels in places where
\NP-hardness fails to do so.
To achieve this, parameter complexity introduces a notion of parameterization of the input instance, where
the parameter is just some secondary measurement of the input instance.
Such parameter is then used together with the input instance size when describing a computational complexity of \NPh problems.
Given a problem, there is typically many parameters we can parameterize the problem by.
This leads us to the importance of choosing the right parameter.

When parameterizing some problem, we typically look for parameters which are small in real-world applications,
as one of the main goal when studying parameterized problems is to find algorithms for these problems in which
the running time is exponential only in the parameter, leaving the running time polynomial in the, potentially enormous, input size.
However, not every choice of the parameter let us design such an efficient algorithm, as it seems that many problems with
certain parameters admit no efficient algorithm at all.
That means the hardness, or tractability, of the problem depends on the parameter we use.
Typically, the more information a parameter carries about the input instance, the higher are chances for us to exploit this
information and design a faster algorithm.

The typical parameter we often try is the size, or other property, of the solution we are looking for.
The other type of parameter is a measure of some property of the input instance.
For example the maximum degree, regularity or treewidth measurings of the input graph.
For non-graph instances, it may be the maximum length of a string when the input instance consists of a set of strings,
or a number of variables when working with Boolean formulas.

Given problem parameterized by $p$, algorithm designers commonly talk about some function of $p$, $f(p)$.
Let us note that it is also possible to use more than one parameter.
Having parameters $k$ and $l$, we then talk about a function of $k$ and $l$, $f(k,l)$.
However, we can (and we actually do) express the parameterization by $k$ and $l$ by using just one parameter $k+l$, $f(k+l)$.

The two most fundamental complexity classes of this theory, used to reason about the complexity of parameterized problems,
are \FPT and \XP.
To distinguish between \NPh problems that are in \XP but not in \FPT, we can use another set of complexity classes, the \W-hierarchy.
There is also a complexity class called \pNP.
What follows are the formal definitions of some of these classes, together with definitions of parameterized problem
and parameterized reduction.
The hierarchy of the classes can be seen in Figure \ref{fig:complexityClasses}.


% Figure - complexity classes
\begin{figure}[t]
    \centering
    \begin{tikzpicture}
        \pgftransformscale{.8}
        %% Horizontal bar
        \draw[very thick] (6,0) -- (-6,0);
        % FPT
        \draw (-2,0) parabola bend (0,2) (2,0);
        \node at (0,1) {\FPT};
        % W[1]
        \draw (-2.5,0) parabola bend (0,3) (2.5,0);
        \node at (0,2.5) {\textsf{W}[1]};
        % W[2]
        \draw (-3,0) parabola bend (0,4) (3,0);
        \node at (0,3.5) {\textsf{W}[2]};
        % % W[t]
        \draw (-4,0) parabola bend (0,6) (4,0);
        \node[anchor=north] at (0,5.8) {
            \begin{tabular}{c}
                \textsf{W}[$t$] \\
                $\vdots$ \\
            \end{tabular}
        };
        % XP
        \draw[rounded corners] (-6,0) .. controls (-1,8) and (9,17) .. (6,0);
        \node at (4,8) {\XP};
        % para-NP
        \draw[rounded corners] (6,0) .. controls (1,8) and (-9,17) .. (-6,0);
        \node at (-4,8) {\pNP};
    \end{tikzpicture}
    \caption
    [Relations among the parameterized complexity classes]
    {
        Relations among the parameterized complexity classes.
        Inspired by the depiction from Flum and Grohe \cite[p.~97]{Flum2006}.
    }
    \label{fig:complexityClasses}
\end{figure}


% Parameterized problem
\begin{definition}[Parameterized problem]
    A parameterized problem is a language $L \subseteq \Sigma^* \times \mathbb{N}$, where
    $\Sigma$ is a fixed, finite alphabet and $\Sigma^*$ is a set of all strings over $\Sigma$.
    For an instance $(x, k) \in \Sigma^* \times \mathbb{N}$, $k$ is called the parameter.
\end{definition}

One particular problem parameterized by different parameters leads to different parameterized problems.

% Parameterized reduction
\begin{definition}[Parameterized reduction]
    Let $A,B \subseteq \Sigma^* \times \mathbb{N}$ be two parameterized problems.
    A parameterized reduction from $A$ to $B$ is an algorithm that, given an instance $(x, k)$ of $A$,
    outputs an instance $(x', k')$ of $B$ such that
    \begin{itemize}
        \item $(x, k)$ is a $yes$-instance of $A$ if and only if $(x', k')$ is a $yes$-instance of $B$,
        \item $k' \leq g(k)$ for some computable, nondecreasing function $g$, and
        \item the running time is $f(k) \cdot |x|^{\mathcal{O}(1)}$ for some computable, nondecreasing function $f$.
    \end{itemize}
\end{definition}

% FPT problem and class
\begin{definition}[FPT]
    A parameterized problem $L$ is called fixed-parameter tractable (FPT) if there exists
    \begin{itemize}
        \item an algorithm $\mathcal{A}$, called fixed-parameter algorithm, or FPT algorithm,
        \item a computable, nondecreasing function $f : \mathbb{N} \times \mathbb{N}$
        \item and a constant $c$
    \end{itemize}
    such that, given $(x,k) \in \Sigma^* \times \mathbb{N}$,
    the algorithm $\mathcal{A}$ correctly decides whether $(x, k) \in L$ in time bounded by
    $f(k) \cdot |(x,k)|^c$.
\end{definition}

The complexity class containing all fixed-parameter tractable problems is called \FPT.
An example of a problem from \FPT is $n$-variable \textsc{SAT} parameterized by $n$.
Indeed, the problem can be solved in time $\Theta(2^n)$ by simply trying each of the $2^n$
possible evaluations for the variables and checking if the evaluation is correct in time $\Theta(n)$.
Moreover, we know from the Exponential Time Hypothesis, formulated by Impagliazzo and Paturi \cite{Impagliazzo1999},
that the problem cannot be solved in time $2^{o(n)}$

The typical goal when designing FPT algorithms is to make factor $f(k)$ and constant exponent~$c$
in the running time boundary as small as possible.
We can see that if the parameter is equal to the input size, then \FPT become exponential and,
on the other hand if $k = 1$, then \FPT become~\Po~\cite{Koutensky2020}.

The only difference between FPT and para-NP problems is that algorithms for para-NP problems are
nondeterministic in general.
We do not provide the exact definition of para-NP problems as they are not important to us.
From the \pNP-hardness perspective, we can say that a problem is \pNPh if it is \NPh already for a constant value of the parameter.

% XP
\begin{definition}[XP]
    A parameterized problem $L$ is called slice-wise polynomial (XP) if there exists
    \begin{itemize}
        \item an algorithm $\mathcal{A}$
        \item and two computable, nondecreasing functions $f,g : \mathbb{N} \times \mathbb{N}$
    \end{itemize}
    such that, given $(x,k) \in \Sigma^* \times \mathbb{N}$,
    the algorithm $\mathcal{A}$ correctly decides whether $(x, k) \in L$ in time bounded by
    $f(k) \cdot |(x,k)|^{g(k)}$.
\end{definition}

The complexity class containing all slice-wise polynomial problems is called \XP.
An example of a problem from \XP is $k$-\textsc{Independent Set}.

Note that definitions of parameterized problems FTP and XP differs only in a running time boundary of algorithm $\mathcal{A}$
and so $\FPT \subseteq \XP$.
Although both FPT and XP algorithms run in polynomial time for every fixed value of the parameter,
the underlying difference between them is that FPT algorithms have the combinatorial explosion
restricticted to the parameter only, leaving the exponent of the instance size constant, in other words,
FPT algorithms are more efficient than XP algorithms.


\subsection{\W-hierarchy}

As stated by Cygan et al. \cite[p.~423]{Cygan2015},
there are thousands of natural problems which are \NPc and which are reducible to each other,
meaning that in this sense, they are equally hard and thus we can say that
they occupy the same level of hardness.
However, we cannot say the same when talking about parameterized problems as it seems there are
different levels of hardness for such problems and, in this sense, even basic problems
seem to be differently hard as they occupy different hardness levels.
For this reason, \W-hierarchy was introduced by Downey and Fellows \cite{Downey1999} as
an attempt to shed a light on these apparent differences in hardness of parameterized problems. 

The levels of \W-hierarchy are marked as \textsf{W}[$t$] for $t \in \mathbb{N} \wedge t \geq 0$,
where each level represents its own complexity class.
The most important level for us will be \Wone,
as well as the fact that the $k$-\textsc{Clique} problem is \Wone-complete \cite{Downey1999}.

We do not provide the exact definition and further description of \W-hierarchy
because it is not important for the purposes of this work.
On the other hand, what is important is a fact that we interpret the \W-hardness as an evidence that
a problem is not fixed-parameter tractable.
This interpretation is based on a general assumption that $\FPT \neq \Wone$.
To show that certain problem is \Wh, it is typically shown a parameterized reduction from some other problem,
which is already known to be \Wh, to it.

%---------------------------------------------------------------
\chapter{Problem statement}\label{chapter:ProblemStatement}
%---------------------------------------------------------------

Now we define the central problem of this thesis, the \HL problem.
We also provide a detailed description of the problem, a motivation behind its formulation,
an explanation of its hardness, an introduction to notation we use and a presentation of results from the literature. 

\vspace{2.5mm}
\begin{tcolorbox}[
    enhanced, skin=enhancedmiddle,
    arc=0pt, outer arc=0pt,
    frame hidden, % colframe=decoration,
    %interior hidden,
    colback=backgroundgray!10,
    % top=5mm, bottom=5mm,
    boxsep=0mm,
    borderline={0.75mm}{0mm}{decoration}, borderline={0.75mm}{0.75mm}{backgroundgray},
    % borderline north={1mm}{0mm}{decoration}, borderline north={0.75mm}{0.75mm}{backgroundgray},
    % borderline south={1mm}{0mm}{decoration}, borderline south={0.75mm}{0.75mm}{backgroundgray},
]
\begin{definition}[\HL]\label{def:HL}
    Let $(G, L, b, c, d)$ be the problem instance, then
    \begin{itemize}
        \item $G = (V, E)$ is a network,
        \item $L \subseteq V$ are leaders and the remaining vertices $F = V \setminus L$ are followers,
        \item $c : G \times V \rightarrow \mathbb{R}$ is a centrality measure,
        \item $b$ is a maximum number of edges that we are allowed to add in $G$,
        \item $d$ is a safety margin -- a number of followers whose final centrality should be at least as high as of any leader.
    \end{itemize}
    Given this instance, the goal is to identify a set of maximum $b$ edges between followers W $\subseteq$ F {\texttimes} F
    such that the resulting network $G' = (V, E \cup |W|)$ contains at least $d$ followers $F' \subseteq F$
    whose centrality must be at least as high as the centrality of any leader, that is
    $$|W| \leq b$$
    and
    $$\exists_{F' \subseteq F} |F'| \geq d \wedge \forall_{f \in F'} \forall_{l \in L} c(G', f) \geq c(G', l).$$
\end{definition}
\end{tcolorbox}
\vspace{2.5mm}

In other words, we want to \emph{``identify a set of edges to be added between the followers so that
the ranking of the leaders (based on some centrality measure) drops below a certain threshold''} \cite{Waniek2017}.
Adding such set of edges into given network let leaders to increase their security while not affecting their
influence in the network, although, new members with great influence may arise.

We survey \HLshort only with respect to the degree centrality measure, that is, with $c = c_{deg}$.
We denote this version of the problem as \HLdeg.
We also use symbol $\lambda$ to denote a maximum degree among leaders, this is the minimum degree that
all followers from $F'$ have to reach.
Lastly, we use symbol $\hat{A}$ to denote a set of edges between followers that are not present in $G$ and thus can be added,
$\hat{A} = E(\overline{G[F]})$.

Among the first authors who surveyed the topic of evading social network analysis tools, rather than developing the new ones,
were Waniek et al. \cite[y.~2016]{Waniek2016}.
This topic was further researched by the same authors in the subsequent work \cite[y.~2017]{Waniek2017}
in which the \HL problem was first formulated.
Note that the paper from 2017 was only a preliminary version of the work and the completed paper was
published later in 2021 \cite[y.~2021]{Waniek2021}.

There are two definitions of the \HL problem in the literature and they differ from each other in some details.
Namely, the definition\footnote{
    There is actually also a generalized form of this definition in Waniek's dissertation thesis \cite{WaniekPhD2017},
    where he also consider a set of edges that can be added and a set of edges that can be removed.
    However, we consider the first mentioned set to always contain all possible edges and the second mentioned set
    to be empty, which is also the typical situation in the literature.
    For this reason, we do not distinguish between these two definitions.
}
from Waniek et al. \cite{Waniek2017} and the definition from Dey et al. \cite{Dey2019}.
The definition from Waniek et al. \cite{Waniek2017} allows leaders $L$ to be equal to $V$, $L \subseteq V$,
which is a small difference since for $L = V$ the problem is trivial, but the more important difference is that
it requires centrality measure of followers $F'$ to be strictly greater than that of any leader,
$\forall_{f \in F'} \forall_{l \in L} c(G', f) > c(G', l)$.
We decided to stick to the definition from Dey \cite{Dey2019} because it seems more natural to us to let followers
to be part of a solution as soon as their centrality is at least the same as of all the leaders --
if no leader has centrality measure greater than any follower from $F'$, then
we consider the leaders hidden and safe from detection.
On the other hand, the name ``\HL'' comes from the original definition by Waniek et al. as
the name is more indicative of the possibility that there are many leaders in the network.
Later in Section \ref{section:proofRevision}, we show that complexity results for degree centrality from Waniek et al.
are correct even with this slighty different definition of \HLshort.
Note that there is also a verion of \HL called \textsc{Minimum Hiding Leaders} \cite{Waniek2021}
in which is no budget specified as the goal is to find the smallest possible set of connections between the followers
sufficient to solve the problem.

The motivation behind only adding edges, with no deletion involved, is that we want leaders to maintain
their existing influence in the network.
Indeed, deleting edges incident to leaders would decrease their number of connections and thus potentially
reducing their reach within network.
Also, deleting edges between followers would only decrease the values of all degree, betweenness and closeness
centrality measures for affected followers \cite{Waniek2016},
which is the exact opposite of what we try to achieve in the \HL problem.
From this we can see why it makes sense to only allow adding edges into given network --
because deleting edges incident with leaders violates our requirements on keeping influence of leaders,
whereas deleting edges incident with followers violates our requirements on making leaders more hidden.

Also note that decreasing the value of degree centrality of any given member is a straightforward task
(unlike decreasing other centralities) as it only consists in cutting edges \cite{Waniek2016}.
On the other hand, \HLdeg is \NPc \cite{Waniek2017}.
The core difference between these two problems is that \HLdeg, and \HLshort in general,
is not about decreasing centrality of given member but rather about decreasing their ranking,
the relative position among other nodes with respect to the centrality measure.
To decrease member's ranking, we must increase centrality of some other members.
This fact, together with budget constraint $b$ and safety margin constraint $d$,
is what stands behind the hardness of the \HL problem, because,
as Waniek et al. \cite{Waniek2017} shown, and as we will also show in Chapter \ref{chapter:contribution},
there exist instances of \HLshort where finding a solution corresponds to solving certain \NPh problem.


%---------------------------------------------------------------
\section{Results from the literature}

Here we present literature results for the \HL problem.
We first take a look on results for the degree centrality measure as it is our main concern, then we
complement it by results for other centralities.
For a presentation of our results, please refer fo Section \ref{section:OurResults}.


% degree centrality
\subsection{Results for degree centrality}\label{subsection:ResultsDegree}

Waniek et al. \cite{Waniek2017} show that the \HL problem is \NPc for the degree centrality.
In more detail, they show that the problem is \Wh when parameterized by $b+d$.
We review this proof for our definition of \HLshort in Section \ref{section:proofRevision} and present
a similar result as Theorem \ref{theorem:DB}.

From the work of Dey et al. \cite{Dey2019}, we know that \HLdeg is polynomial time solvable
if the degree of every leader is bounded by some constant.
That means that \HLdeg admits a FPT algorithm when parameterized by degree of leaders, or,
in other words, when parameterized by $\lambda$.
In addition to this, the authors present a $2$-approximation algorithm for \HLdeg which optimizes
the number of edges added.
They empirically evaluate the algorithm in synthetic networks and conclude that it produces near
optimal solutions in practice.
They complement that by proving that if there exists a $(2-\epsilon)$-approximation algorithm
for the above problem for any constant $0 < \epsilon < 1$, then there exists
a $(\frac{\epsilon}{2})$-approximation algorithm for the \textsc{Densest} $k$-\textsc{Subgraph} problem.


% other centralities
\subsection{Results for other centralities}

Waniek et al. next show that \HL is \NPc for the closeness \cite{Waniek2017} and betweenness \cite{Waniek2021} centralities.
In more detail, they show that the problem is \Wh when parameterized by $b$ for both the closeness and the betweenness centralities.
In addition to this, the authors show that \HL cannot be approximated within a ratio of $(1 - \epsilon ) \cdot ln(|F|)$
for any $\epsilon > 0$, unless \Po = \NP, with respect to both the closeness and the betweenness centralities. \todo{is this right?}

For the core centrality measure, Dey et al. \cite{Dey2019} next show that \HLshort is \NPc even if core centrality of
every leader is exactly 3. Meaning that \HLshort is \pNPh when parameterized by the maximum core centrality of leaders.
In addition to this, the authors prove that \HLshort is polynomial time solvable
if the core centrality of every leader is at most 1.
They complement that by proving that there does not exist any $((1 - \alpha) \cdot ln(n))$-approximation algorithm
for any constant $\alpha \in (0, 1)$ which optimizes the number of edges that one needs to add even
when the core centrality of every leader is 3.
 
%---------------------------------------------------------------
\chapter{Contribution}
%---------------------------------------------------------------

\todo{chapter intro}


%---------------------------------------------------------------
\section{Proofs revision}\label{sec:proofRevs}

Waniek et al. \cite{Waniek2017} proposed three hardness results for the \HL problem given degree, closeness and betweenness
centralities. In this section, we review the first aforementioned result with respect to our definition of \HLshort.

% Theorem - degree
\begin{theorem}
    The \HL problem is NP-complete given the degree centrality.
\end{theorem}
\begin{proof}
    \itshape{``
    \todo{change} The problem is trivially in NP, since after the addition of a given [|W|] it is possible to
    compute the degree centrality for all nodes in polynomial time.
    
    Next, we prove that the problem is NP-hard. To this end, we propose a reduction from the NP-complete problem of Finding k-clique.
    The decision version of this problem is defined by a network,
    $G = (V , E)$, and a constant, $k \in \mathbb{N}$, where the goal is to determine whether there exist $k$ nodes in $G$ that form a clique.
    
    Let us assume that $k \geq 3$ (if $k = 2$ then the problem is trivial). Given an instance of the problem
    of Finding k-clique, defined by some $k \geq 3$ and a network $G = (V , E)$, let us construct a network,
    $H = (V', E')$, as follows:
    
    \begin{description}
        \item[The set of nodes:] For every node, $v_i \in V$, we create a single node, $v_i$ , as well as $|N_G(v_i)|$
        other nodes, denoted by $X = \{x_{i,1},\dots, x{i,|N_G(v_i)|} \}$. Additionally, we create one node called $y$,
        as well as $\boldsymbol{n + k}$ other nodes, namely, $L' = l_1,\dots, \boldsymbol{l_{n+k}}$;
        \item[The set of edges:] We create an edge between two nodes $v_i, v_k \in V$ if and only if this edge
        was not present in $G$, i.e., $(v_i, v_j ) \in E' \Leftrightarrow (v_i, v_j ) \notin E$.
        Additionally, for every $v_i$ , we create an edge $(v_i , y)$ as well as an edge $(v_i, x_i, j)$ for every $x_i, j$.
        We also create an edge $(l_i, l_j)$ between every pair of nodes $l_i, l_j \in L'$, except for the edge $(l_1, l_2)$.
        Finally, we create two additional edges, $(l_1, y)$ and $(l_2, y)$.
    \end{description}

    An example of such an $H$ network is illustrated in [figure ommited]. Now, consider the following instance
    of the problem of hiding leaders, $(H, L, b, c, [d] )$, where:

    \begin{description}
        \item $H = (V', E')$ is the network we just constructed;
        \item $L = V' \setminus V$;
        \item $b = \frac{k\cdot(k-1)}{2}$;
        \item $c$ is the degree centrality measure;
        \item $[d] = k$. 
    \end{description}

    Next, we reduce the problem of Finding k-cliques in $G$ to the aforementioned instance of Hiding
    Leaders in $H$ . To this end, from the definition of the problem of Hiding Leaders, we know that the
    edges to be added to $H$ must be chosen from $F \times F$. Since in our instance we have:
    $F = V' \setminus L = V' \setminus (V' \setminus V ) = V$, then the edges to be added to $H$ must be chosen
    from $V \times V$ . However, since the edges in $(V \times V ) \setminus E$ are already present in $H$
    (see how $H$ is created), then the edges to be added to $H$ must be chosen from $E$.
    Out of those edges, we need to choose a subset, $[W] \subseteq E$, as a solution to the problem.
    In what follows, we will show that a solution to the above instance of the problem of \HL in $H$ corresponds
    to a solution to the problem of Finding k-clique in $G$.

    First, note that each of the k nodes with the highest degree centrality in $H$ must be a member of
    $L'$. This is because there are more than k nodes in $L'$, each of which has a degree of $\boldsymbol{n+k-1}$, while
    the degree of every node in $V' \setminus L'$ is smaller than $\boldsymbol{n + k - 1}$. Thus, for $[W]$
    to be a solution to the problem of hiding leaders, the addition of $[W]$ to $H$ must increase
    the degree of at least $k$ nodes in $V$ such that each of them has a degree of at least $n + k - 1$
    (note that the addition of $[W]$ only increases the degrees of nodes in $V$, as we already established
    that $[W] \subseteq E$). Now, since in $H$ the degree of every node in $V$ equals $n$
    (because of the way $H$ is created), then to increase the degree of $k$ such nodes to $n + k - 1$, each of them
    must be an end of at least $k - 1$ edges in $[W]$. But, since the budget in our problem instance is
    $\frac{k\cdot(k-1)}{2}$, then the only possible choice of $[W]$ is the one that increases the
    degree of exactly $k$ nodes in $V$ by exactly $k - 1$. If such a choice of $[W]$ is available, then surely
    those $k$ nodes would form a clique in $G$, since all $\frac{k\cdot(k-1)}{2}$ edges in $[W]$ are taken from $G$.
    }''
\end{proof}


%---------------------------------------------------------------
\section{Our results}

The next proof is built on the same idea Waniek et al. \cite{Waniek2017} presented in their proof of
NP-completeness of \HLshort for the degree centrality measure.
The core idea of the proof is that we construct a network in such way that finding a solution $W$ of the \HL problem
correspond to finding a solution of the problem we started with, essentially having the starting network
present in the constructed network in some manner and then finding a solution inside of it.

\begin{theorem}
    \HL parameterized by $b + d$ is W[1]-hard even if $|L| = 1$.
\end{theorem}
\begin{proof}\label{proofDB}
    To proof this theorem, we propose a parameterized reduction from the $k$-\textsc{Clique} problem on regular graphs,
    which is known to be W[1]-hard with respect to $k$ \cite{Mathieson2008}.
    
    Suppose we have a parameterized $k$-\textsc{Clique} instance $(G, k)$, where $|V(G)|=n$, $G$ is a $r$-regular graph and $n-2 \geq r \geq k \geq 3$
    (for $r > n-2$ or $r < k$ or $k < 3$ the problem is trivial).
    Then, we construct a graph $H$ in the following way:
    \begin{enumerate}
        \item Start with an empty graph $H$.
        \item Add all vertices from $G$, i.e., $V(H) \leftarrow V(G)$, mark these vertices as $F_1$.
        \item Add all edges from $\overline{G}$, i.e., $E(H) \leftarrow E(\overline{G})$.
        \item Add an extra vertex $\ell$, i.e., $V(H) \leftarrow V(H) \cup \{\ell\}$.
        \item Add an edge between vertex $\ell$ and $n - r + (k - 1) \eqqcolon \lambda$ arbitrarily chosen vertices $X \subset F_1$,
              i.e., $E(H) \leftarrow E(H) \cup \{ (\ell, x) \mid x \in X \}$.
        \item For each vertex $v \in V(H) \setminus X \eqqcolon Y$, introduce a new vertex $w_v$ and add edge $\{v, w_v\}$, i.e.,
              $V(H) \leftarrow V(H) \cup \{ w_v \mid v \in Y \}$ and $E(H) \leftarrow E(H) \cup \{ (v, w_v) \mid v \in Y \}$,
              mark set of $w_v$ for each $v \in Y$ as $F_2$.
    \end{enumerate}
    An example of such construction can be seen in figure \ref{fig:proofDB}.

    Note that step 5 can always be done because $r \ge k$, so $|F_1| = n > n - r + (k - 1)$
    and since $deg(l) = 0$ (before step 5), there are enough vertices for $\ell$ to connect with.
    Considering $G$ is $r$-regular, its complement, constructed and added into $H$ in steps 2 and 3, must be $(n-r-1)$-regular.
    After connecting all the $F_1$ vertices either with $\ell$, or the corresponding $w_v$ in steps 5 and 6,
    they all end up with degree $n-r$; in other words, graph $H[F_1]$ is $(n-r)$-regular.
    Also note that the construction of $H$ is done in time polynomial to $n$.
    Now we show how finding a $k$-clique in graph $G$ corresponds to finding a solution $W$ of \HLshort in $H$.

    Let us take $ \mathcal{I} = (H, \{\ell\}, \frac{k\cdot(k-1)}{2}, c, k)$, where $c$ is a degree centrality measure,
    as an instance of \HLshort with $k' \coloneqq b + d = \frac{k\cdot(k-1)}{2} + k$ as a parameter.
    First we note that degree of the only leader $\ell$ is $\lambda$, as presented above.
    The other vertices $V(H) - l$ are naturally followers of which we can recognize two types, $F_1$ and $F_2$,
    where $\forall_{f_1 \in F_1} deg(f_1)$ = $n-r$ and $\forall_{f_2 \in F_2} deg(f_2)$ = $1$.
    Whereas $F_1$ are vertices of the original graph $G$,
    $F_2$ plays a role of ``partners'' of vertices $Y$, since their only job is to substitute a missing connection with $\ell$.

    Because we just have one leader in $\mathcal{I}$, it is clear that for any follower $f' \in F'$ applies that
    $deg(f') \geq \lambda = n - r + (k - 1)$.
    Also, because $n > r$, $\max\limits_{\forall f_1 \in F_1}deg(f_1) = n-r > 1 = \max\limits_{\forall f_2 \in F_2}deg(f_2)$.
    With this in mind, we state the following:

    \begin{lemma}\label{lemmaInProof}
        No $F'$ can contain vertex from $F_2$, i.e., $F' \cap F_2 = \emptyset$.
    \end{lemma}
    \begin{subproof}
        Assume $F' \subseteq F_1$. This is a valid assumption because $|F_1| = n > k$ and it must be at least $d = k$ vertices in $F'$.
        Because all vertices from $F_1$ have degree $n-r$, we must have added at least $\lambda - (n - r) = k - 1$ new neighbours to all vertices from $F'$.
        Since the budget $b$ in $\mathcal{I}$ is $\frac{k\cdot(k-1)}{2}$,
        the only possible way of how $F'$ could have been constructed is that $|F'|=k$ and the newly added edges $W \subset F_1 \times F_1$, $|W| = b$
        connect all the $k$ vertices $f' \in F'$ in a way that graph $(F', W)$ forms a $k$-clique.
        From this and from the statements above, we can see that no vertex from $F_2$ can be in $F'$
        since there will never be large enough budget $b$ to let us reach the safety margin $d$.
    \end{subproof}

    As shown in the proof of lemma \ref{lemmaInProof}, every feasible solution $W$ of \HLshort for instance $\mathcal{I}$, together with corresponding $F'$,
    will form a $k$-clique.
    Note that every edge from $W$ must be an edge from the original graph $G$ because $W \subset F_1 \times F_1$ and $F_1 = E(G)$
    and also all the edges from $E(\overline{G})$ are already in $E(H)$ so they cannot be in $W$ since it only contains newly added edges.

    Because a solution $W$ for \HLshort in $H$ exists if and only if a $k$-clique in $G$ exists,
    we can conclude that finding a solution for the \HL problem in graph $H$ is exactly the same as
    finding a solution for the $k$-\textsc{Clique} problem in graph $G$.

    The reduction presented in this proof is a valid parameterized reduction because;
    $(G, k)$ is a yes-instance of the $k$-\textsc{Clique} problem if and only if $\mathcal{I}$ is a yes-instance of \HLshort;
    the construction of $H$ is done in time polynomial to $n$;
    and there is a function of $k$, $g(k) = \frac{k\cdot(k-1)}{2} + k$, upper-bounding the parameter $k'$ of $\mathcal{I}$.
\end{proof}

% Figure
\begin{figure}[t!]
    \centering
    % G
    \begin{subfigure}[t]{.46\textwidth}
        \centering
        \begin{tikzpicture}[node distance={16mm}, scale=0.8, every node/.style={scale=0.8}]
            \tikzstyle{vertex} = [circle, draw=black, text=white]
            \tikzstyle{edge} = []
            
            \node[vertex] (g1) at (0,0) {$f_{1_1}$};
            \node[vertex] (g2) [right of = g1] {$f_{1_2}$};
            \node[vertex] (g3) [below of = g1] {$f_{1_3}$};
            \node[vertex] (g4) [right of = g3] {$f_{1_4}$};
            \node[vertex] (g5) [below of = g3] {$f_{1_5}$};
            \node[vertex] (g6) [right of = g5] {$f_{1_6}$};
            \draw[edge] (g1)--(g2);
            \draw[edge] (g5)--(g6);
            \draw[edge] (g1)--(g3);
            \draw[edge] (g3)--(g5);
            \draw[edge] (g2)--(g4);
            \draw[edge] (g4)--(g6);
            \draw[edge] (g3)--(g2);
            \draw[edge] (g3)--(g6);
            \draw[edge] (g4)--(g1);
            \draw[edge] (g4)--(g5);
            \draw[edge] (g1) to [out=180,in=180,looseness=0.5] (g5);
            \draw[edge] (g2) to [out=0,in=0,looseness=0.5] (g6);
            % dummy to ensure the same height as the other subfig
            \node (l) [right of = g4] {};
            \draw[color=white] (l) to [out=315,in=270] (g5);
        \end{tikzpicture}
        \caption{
            ~A $4$-regular ($r = 4$) graph $G$ with $n=6$ vertices,
            playing a role of the input graph for the $k$-\textsc{Clique} problem and
            is the starting point of our construction.
        }
    \end{subfigure}%
    ~~~~
    % H
    \begin{subfigure}[t]{.46\textwidth}
        \centering
        \begin{tikzpicture}[node distance={16mm}, scale=0.8, every node/.style={scale=0.8}]
            \tikzstyle{followerG} = [circle, draw=black]
            \tikzstyle{edgeG} = [dashed, draw=green!40, thick]
        
            \tikzstyle{edgeCompl} = []

            \tikzstyle{followerF2} = [circle, fill=red!50]
            \tikzstyle{edgeF2} = [draw=red!50]

            \tikzstyle{square} = [regular polygon, regular polygon sides=4]
            \tikzstyle{leader} = [square, fill=blue!50, label={[text=blue!50]above:$\ell$}]
            \tikzstyle{edgel} = [color=blue!50]

            % G in H
            \node[followerG] (g1) at (0,0) {$f_{1_1}$};
            \node[followerG] (g2) [right of = g1] {$f_{1_2}$};
            \node[followerG] (g3) [below of = g1] {$f_{1_3}$};
            \node[followerG] (g4) [right of = g3] {$f_{1_4}$};
            \node[followerG] (g5) [below of = g3] {$f_{1_5}$};
            \node[followerG] (g6) [right of = g5] {$f_{1_6}$};
            \draw[edgeG] (g1)--(g2);
            \draw[edgeG] (g5)--(g6);
            \draw[edgeG] (g1)--(g3);
            \draw[edgeG] (g3)--(g5);
            \draw[edgeG] (g2)--(g4);
            \draw[edgeG] (g4)--(g6);
            \draw[edgeG] (g3)--(g2);
            \draw[edgeG] (g3)--(g6);
            \draw[edgeG] (g4)--(g1);
            \draw[edgeG] (g4)--(g5);
            \draw[edgeG] (g1) to [out=180,in=180,looseness=0.5] (g5);
            \draw[edgeG] (g2) to [out=0,in=0,looseness=0.5] (g6);
            % complement
            \draw[edgeCompl] (g1)--(g6);
            \draw[edgeCompl] (g2)--(g5);
            \draw[edgeCompl] (g3)--(g4);
            % leader
            \node[leader] (l) [right of = g4] {};
            \draw[edgel] (l)--(g4);
            \draw[edgel] (l)--(g6);
            \draw[edgel] (l) to [out=45,in=90] (g1);
            \draw[edgel] (l) to [out=315,in=270] (g5);
            % partners
            \node[followerF2, label={[text=red!50]above:$f_{2_1}$}] (f2) [above of = g2] {};
            \node[followerF2, label={[text=red!50]above:$f_{2_2}$}] (f3) [left of = g3] {};
            \draw[edgeF2] (f2)--(g2);
            \draw[edgeF2] (f3)--(g3);
        \end{tikzpicture}
        \caption{
            ~A $2$-regular graph $H$ constructed from graph $G$.
            The black vertices are followers from $F_1$,
            the red vertices are followers from $F_2$ and
            the blue vertex is leader.
            The black edges are edges $E(\overline{G})$ and the green, dashed lines represent edges from $G$,
            which are not present in $H$, but from which a potential solution $W$ of \HLshort would be picked.
        }
    \end{subfigure}
    \caption{A sample construction of graph $H$ as presented in proof \ref{proofDB}.}
    \label{fig:proofDB}
\end{figure}

% Corollary
Because we have just shown there is the reduction from the $k$-\textsc{Clique} problem on regular graphs to the
\HL problem with a constant number of leaders and because this reduction is done in polynomial time,
we directly get this corollary.

\begin{corollary}
    \HL is para-NP-hard with respect to $|L|$.
\end{corollary}

%---------------------------------------------------------------
\chapter{Conclusion}
%---------------------------------------------------------------

In this thesis we introduced the reader to the topic of covert networks and their analysis,
with emphasis on the \HL problem which we described, motivated its study
and for which we provided an overview of results from the literature.
We pointed out there are multiple definition of the problem in the literature and described our motivation
for picking one,
we then reviewed some of the results from the literature with repsect to this definition.
The results from the literature are described in a language of classical complexity theory,
so we put them into a perspective of the parameterized complexity framework, which we also briefly introduced.
Then, we focused on the problem with respect to the degree centrality measure
and presented our own parameterized complexity results in this domain.
Last, we put together both our results and results from the literature for the degree centrality
and visualize them in an overviewing graph.

Some of the future work may involve revision of more proofs from the literature with respect to our definition,
or inspecting the problem with different sets of parameters in consideration.
To this end, a parameter for which \HL may admit an FPT algorithm in some cases seems to be a \emph{mininum vertex cover}.

%\include{text/samples/cviko.tex}
%\include{text/samples/text.tex}

\appendix\appendixinit % do not remove these two commands

% \chapter{Some appendix}


Sem přijde to, co nepatří do hlavní části.
 % include `appendix.tex' from `text/' subdirectory

\backmatter % do not remove this command

\printbibliography % print out the BibLaTeX-generated bibliography list

% \include{text/medium} % include `medium.tex' from `text/' subdirectory

\end{document}
