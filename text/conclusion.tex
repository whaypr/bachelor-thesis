%---------------------------------------------------------------
\chapter{Conclusion and future work}
%---------------------------------------------------------------

In this thesis, we introduced the reader to the topic of covert networks and their analysis,
with emphasis on the \HL problem for which we provided a detailed description, we motivated its study
and explained its hardness.
We mostly followed up on the work from Waniek et al. and Dey with Medya \cite{Waniek2017,Dey2019}.
We pointed out the authors use definitions that are different from each other and described our motivation
for picking the one from Dey and Medya.
We then reviewed the result for the degree centrality from Waniek et al. with respect to our definition.

The results from the literature are described in a language of classical complexity theory,
so we put them into a perspective of the parameterized complexity framework, which we also briefly introduced.
We then focused on the problem with respect to the degree centrality measure
and presented our own parameterized complexity results in this domain.
Namely, we showed that \HL is \Wh when parameterized by $b+d$ even if $|L| = 1$.
Because the reduction in the proof was done in polynomial time,
we got that \HLshort is \pNPh with respect to $|L|$.
We then showed that \HLshort is in \XP when parameterized by $b$ or $d$.
Last, we put together both our results and results from the literature for the degree centrality
and parameters $|L|$, $b$, $d$, $\lambda$ and visualized them in an overviewing graph.

Some of the future work on the topic of the \HL problem may involve
revisioning the results for the closeness and betweenness centralities from Waniek et al. \cite{Waniek2021full}
with respect to our definition,
or inspecting the problem for various combinations of centralities and parameters for which there are no results yet.

For instance, it would be interesting to see some parameterized complexity results for the core centrality
as the problem seems much harder for the core centrality than for the degree centrality \cite{Dey2019}.
For the variant with the core centrality, one could also survey the computational complexity of \HLshort
when the core centrality of every leader is at most 2,
as Dey and Medya \cite{Dey2019} inspect only the situation where the core centrality of every leader is exactly 3.

For the variant with the degree centrality, which we studied the most in this thesis,
it would be interesting to see the parameterization which gives rise to an FPT algorithm
as there is known no such algorithm yet.
It seems that the parameterization by the \emph{vertex cover number} could, in some cases,
bring promising results in this direction.
Vertex cover number is the size of a \emph{minimum vertex cover},
minimum vertex cover is the smallest possible number of vertices $V$ of a given graph $G$ such that each edge from $E(G)$ is incident
to at least one vertex from $V$.
For the parameter $\lambda$, \HLdeg is in \XP and
we think that it is also \Wh, so, another immediate future work is to prove or disprove this conjecture.

The \HL problem has not yet been studied parameterized by some structural limitation of the input graph.
One such structural limitation is the aforementioned \emph{minimum vertex cover}.
Another interesting structural parameter to expose the problem to is a \emph{treewidth}.
Treewidth is, roughly speaking, a measurement of how well a graph can be decomposed into pieces that are connected in a
tree-like fashion \cite[p.~151]{Cygan2015}.
The notion of treewidth was described by Robertson and Seymour in the series of papers \cite{Robertson1984,Robertson1986.2,Robertson1986.5}. 
Treewidth has vast applications in the design of parameterized graph algorithms and
in the design of graph algorithms in general.

Another open problem in this topic is exploring the average case computational complexity of \HL for various centrality measures,
rather than doing the worst-case analysis.
The results from the literature show that the problem is intractable only in the worst case so
some heuristics may exist that efficiently solve most of the instances typical in some real-world application.
If true, then the apparent complexity of hiding in networks by manipulating their structure
could be overcome in many real-world cases.~\cite{Dey2019}
