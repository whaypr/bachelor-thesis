%---------------------------------------------------------------
\chapter{Preliminaries}
%---------------------------------------------------------------

In this chapter, we introduce the reader to the theory needed for a comprehensive understanding of this thesis.


%---------------------------------------------------------------
\section{Basic graph theory concepts}

Let us start with basic concepts from a graph theory.
Definitions in this section are based on the text from the graph theory textbook from Diestel \cite{Diestel2018}.
\todo{check}

% Graph
\begin{definition}[Graph]
    Graph $G$ is an ordered pair of sets $V$ and $E$, $G=(V,E)$.
    Set $V$ is a finite set of arbitrary items called vertices (or nodes).
    Set $E$ is a finite set of unordered pairs of vertices called edges, $E \subseteq \{\{x,y\} \mid x,y \in V \wedge x \neq y\}$.
    Sets of all vertices $V$ and edges $E$ of graph $G$ are in general denoted by $V(G)$ and $E(G)$ respectively.
\end{definition}
We use a standard graph definition, where graphs are undirected (edges are not ordered),
unweighted (edges has uniformed ``value''), without loops (edges are pairs of distinct vertices) and
without multiple edges (edges are unique pairs).
These graphs are called \emph{simple}.

% Degree
\begin{definition}[Vertex degree]
    Vertex degree (or degree) of vertex $v$ in some graph $G$ is the number of edges that are incident
    (there exist another vertex $w$ that $\{v,w\} \in E(G)$) with $v$.
    In other words degree is the number of edges vertex $v$ is a part of.
\end{definition}

% r-regular graph
\begin{definition}[$r$-regular graph]
    Graph $G$ is $r$-regular if degree of each vertex from $V(G)$ is $r$, i.e., $(\forall v \in V(G))(deg(v) = r)$.
\end{definition}

% Regular graph
\begin{definition}[Regular graph]
    Graph $G$ is regular if there exists some $r \in \mathbb{N}$ for which $G$ is $r$-regular.
\end{definition}

% Complement graph
\begin{definition}[Complement graph]
    Complement graph $\overline{G}$ of graph $G$ is the graph on vertices $V(G)$,
    where two vertices are adjacent (there is an edge between them) if and only if
    they are not adjacent in $G$, i.e., $(\forall e \in V(G) \times V(G))(e \in E(\overline{G}) \Leftrightarrow e \notin E(G))$.
\end{definition}

% Induced subgraph
\begin{definition}[Induced subgraph]
    Induced subgraph $G[S]$ of graph $G$ is the graph on vertices $V(G[S]) = S \subset V(G)$,
    where two vertices are adjacent if and only if they are adjacent in $G$, i.e.,
    $(\forall u,v \in S)(\{u,v\} \in E(G[S]) \Leftrightarrow \{u,v\} \in E(G))$.
\end{definition}

% Complete graph
\begin{definition}[Complete graph]
    Complete graph is the graph $G$ if there is an edge between every pair of vertices from $V(G)$, i.e.,
    $(\forall u,v \in V(G))(\{u,v\} \in E(G))$.
\end{definition}

% Clique
\begin{definition}[Clique]
    Clique $C$ is the subgraph of graph $G$ where $G[E(C)]$ is a complete graph.
\end{definition}

% Vertex cover
\begin{definition}[Vertex cover]
    Vertex cover of graph $G$ is the subset $S \subseteq V(G)$ where each edge from $E(G)$ is
    covered by some vertex from $S$. In other words, at least one endpoint of each edge is present in $S$, i.e.,
    $(\forall \{u,v\} \in E(G))(u \in S \vee v \in S)$.
\end{definition}


\subsection{Graph problems}

This subsection provides definitions of graph problems referenced later in the text.

% k-Clique
\begin{definition}[$k$-\textsc{Clique}]
    Given graph $G$, the $k$-\textsc{Clique} problem is to determine if
    there exists a clique $C$ in $G$ where $|V(C)| = k$.
\end{definition}
There are many other graph problems related to cliques. The $k$-\textsc{Clique} problem is a decision problem but
there are also optimization variants, for example the \textsc{Maximum Clique} problem.

% Minimum Vertex Cover
\begin{definition}[\textsc{Minimum Vertex Cover}]
    Given graph $G$, the \textsc{Minimum Vertex Cover} problem is to find
    the vertex cover $M$ that there is no other vertex cover $N$ with fewer vertices than $M$, i.e., 
    $(\nexists N)(|N| < |M|)$.
\end{definition}
Do not confuse \emph{minimum vertex cover} from the \textsc{Minimum Vertex Cover} problem defined above with \emph{minimal vertex cover},
which is the vertex cover such that any of its non-trivial subsets is not a vertex cover.
Every \emph{minimum vertex cover} is also \emph{minimal vertex cover} but not necessarily vice versa.


%---------------------------------------------------------------
\section{Parameterized complexity}

We use parameterized complexity definitions as they are stated by
Garey with Johnson \cite{Garey1990} and Cygan et al. \cite{Cygan2015}.
\todo{revide}

\begin{definition}[Parameterized problem]
    A parameterized problem is a language $L \subseteq \Sigma^* \times \mathbb{N}$, where
    $\Sigma$ is a fixed, finite alphabet. For an instance $(x, k) \in \Sigma^* \times \mathbb{N}$, $k$ is called the parameter.
\end{definition}


%---------------------------------------------------------------
\section{Centrality measures}

\todo{Centrality measures}

\begin{definition}[Degree centrality]
    A degree centrality, introduced by Shaw \cite{Shaw1954}, measures an importance of a vertex by its degree
    (a number of its neighbours).
    A degree centrality of the vertex $v$ in network $G$, normalized by size of the network, is defined as:
    $$c_{deg}(G, v) = \frac{deg(v)}{|V(G) - 1|}.$$
\end{definition}