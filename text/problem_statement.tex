%---------------------------------------------------------------
\chapter{Problem statement and analysis}\label{chapter:ProblemStatement}
%---------------------------------------------------------------

Now we define the central problem of this thesis, the \HL problem.
In this chapter, we also provide an introduction to the notation we use,
a detailed description of the problem, a motivation behind its formulation, an explanation of its hardness
and a review of the literature on this topic with a description of the differences between problem definitions
and with a presentation of the literature results.


\vspace{2.5mm}
\begin{tcolorbox}[
    enhanced, skin=enhancedmiddle,
    arc=0pt, outer arc=0pt,
    frame hidden, % colframe=decoration,
    %interior hidden,
    colback=backgroundgray!10,
    % top=5mm, bottom=5mm,
    boxsep=0mm,
    borderline={0.75mm}{0mm}{decoration}, borderline={0.75mm}{0.75mm}{backgroundgray},
    % borderline north={1mm}{0mm}{decoration}, borderline north={0.75mm}{0.75mm}{backgroundgray},
    % borderline south={1mm}{0mm}{decoration}, borderline south={0.75mm}{0.75mm}{backgroundgray},
]
\begin{definition}[\HL]\label{def:HL}
    Let $(G, L, b, c, d)$ be the problem instance, then
    \begin{itemize}
        \item $G = (V, E)$ is a network,
        \item $L \subseteq V$ are leaders and the remaining vertices $F = V \setminus L$ are followers,
        \item $b$ is the maximum number of edges that we are allowed to add in $G$,
        \item $c : G \times V \rightarrow \mathbb{R}$ is a centrality measure,
        \item $d$ is a safety margin -- the number of followers whose final centrality should be at least as high as of any leader.
    \end{itemize}
    Given this instance, the goal is to identify a set of maximum $b$ edges between followers W $\subseteq$ F {\texttimes} F
    such that the resulting network $G' = (V, E \cup W)$ contains at least $d$ followers $F' \subseteq F$
    whose centrality must be at least as high as the centrality of any leader, that is,
    $$|W| \leq b$$
    and
    $$\exists_{F' \subseteq F} |F'| \geq d \wedge \forall_{f \in F'} \forall_{l \in L} c(G', f) \geq c(G', l).$$
\end{definition}
\end{tcolorbox}
\vspace{2.5mm}

In other words, we want to \emph{``identify a set of edges to be added between the followers so that
the ranking of the leaders (based on some centrality measure) drops below a certain threshold''}~\cite{Waniek2017}.
Adding such a set of edges into a given network lets leaders increase their security while not affecting their
influence in the network, although, new members with great influence may arise.

We survey \HLshort only with respect to the degree centrality measure, that is, with $c = c_{deg}$.
We denote this version of the problem as \HLdeg.
We also use symbol $\lambda$ to denote a maximum degree among leaders, this is the minimum degree that
all followers from $F'$ have to reach.
Lastly, we use symbol $\hat{A}$ to denote a set of edges between followers that are not present in $G$ and thus can be added,
$\hat{A} = E(\overline{G[F]})$.

Among the first authors who surveyed the topic of evading social network analysis tools, rather than developing new ones,
were Waniek et al. \cite{Waniek2016}.
This topic was further researched by the same authors in the subsequent work \cite{Waniek2017}
in which the \HL problem was first formulated.
Note that the paper from 2017 was only a preliminary version of the work and the completed paper was
published later in 2021 \cite{Waniek2021full}.
There was also formulated an optimization variant of the \HL problem in the completed paper, called \textsc{Minimum Hiding Leaders}, 
in which there is no budget specified, as the goal is to find the smallest possible set of connections between the followers
sufficient to solve the problem.
In another work from 2021~\cite{Waniek2021}, Waniek et al. studied a problem similar to the \HL problem,
which is called \textsc{Local Hiding}.
The \HL problem was then also studied by Dey and Medya \cite{Dey2019}.


There are two definitions of the \HL problem in the literature and they differ from each other in some details.
Namely, the definition\footnote{
    There is actually also a generalized form of this definition in Waniek's dissertation thesis \cite{WaniekPhD2017},
    similar to the definition of \textsc{Local Hiding},
    where he also considers a set of edges that can be added and a set of edges that can be removed.
    However, we consider the first-mentioned set to always contain all possible edges and the second-mentioned set
    to be empty, which is also the typical situation in the literature.
    For this reason, we do not distinguish between these two definitions.
}
from Waniek et al. \cite{Waniek2017} and the definition from Dey and Medya \cite{Dey2019}.
The definition from Waniek et al. \cite{Waniek2017} allows leaders $L$ to be equal to $V$, $L \subseteq V$,
which is a small difference since for $L = V$ the problem is trivial, but the more important difference is that
it requires the centrality measure of followers $F'$ to be strictly greater than that of any leader,
$\forall_{f \in F'} \forall_{l \in L} c(G', f) > c(G', l)$.
We decided to stick to the definition from Dey and Medya \cite{Dey2019} because it seems more natural to us to let followers
to be part of a solution as soon as their centrality is at least the same as of all the leaders --
if no leader has a centrality measure greater than any follower from $F'$, then
we consider the leaders hidden and safe from detection.
On the other hand, the name ``\HL'' comes from the original definition by Waniek et al. as
the name is more indicative of the possibility that there are many leaders in the network.
Later in Section \ref{section:proofRevision}, we show that complexity results for degree centrality
from Waniek et al. \cite{Waniek2017} still hold even with this slightly different definition of \HLshort.

The motivation behind only adding edges, with no deletion involved, is that we want leaders to maintain
their existing influence in the network.
Indeed, deleting edges incident to leaders would decrease their number of connections and thus potentially
reduce their reach within the network.
Also, deleting edges between followers would only decrease the values of all degree, betweenness and closeness
centrality measures for affected followers \cite{Waniek2016},
which is the exact opposite of what we try to achieve in the \HL problem.
From this, we can see why it makes sense to only allow adding edges into a given network --
because deleting edges incident with leaders violates our requirements on keeping the influence of leaders,
whereas deleting edges incident with followers violates our requirements on making leaders more hidden.

Also, note that decreasing the value of the degree centrality of any given member is a straightforward task
(unlike decreasing other centralities) as it only consists in cutting edges \cite{Waniek2016}.
On the other hand, \HLdeg is \NPc \cite{Waniek2017}.
The core difference between these two problems is that \HLdeg, and \HLshort in general,
is not about decreasing the centrality value of a given member but rather about decreasing their ranking,
the relative position among other nodes with respect to the centrality measure.
To decrease member's ranking, we must increase the centrality of some other members.
This fact, together with budget constraint $b$ and safety margin constraint $d$,
is what stands behind the hardness of the \HL problem, because,
as Waniek et al. \cite{Waniek2017} shown, and as we will also show in Chapter \ref{chapter:contribution},
there exist instances of \HLshort where finding a solution corresponds to solving certain \NPh problem.


%---------------------------------------------------------------
\section{State of the art}

Here we present literature results for the \HL problem.
We first take a look at results for the degree centrality measure as it is our main concern, then we
complement it with results for other centralities.
For a presentation of our results, please refer to Section \ref{section:OurResults}.


% degree centrality
\subsection{Results for degree centrality}\label{subsection:ResultsDegree}

Waniek et al. \cite{Waniek2017} show that the \HL problem is \NPc for the degree centrality.
In more detail, they show that the problem is \Wh when parameterized by $b+d$.
We review this proof for our definition of \HLshort in Section \ref{section:proofRevision} and present
a similar result as Theorem~\ref{theorem:DB}.

From the work of Dey and Medya \cite{Dey2019}, we know that \HLdeg is polynomial-time solvable
if the degree of every leader is bounded by some constant.
That means that \HLdeg admits an XP algorithm when parameterized by the degree of leaders, or,
in other words, when parameterized by $\lambda$.
In addition to this, the authors present a $2$-approximation algorithm for \HLdeg which optimizes
the number of edges added.
They empirically evaluate the algorithm in synthetic networks and conclude that it produces near
optimal solutions in practice.
They complement that by proving that if there exists a $(2-\epsilon)$-approximation algorithm
for the above problem for any constant $0 < \epsilon < 1$, then there exists
a $(\frac{\epsilon}{2})$-approximation algorithm for the \textsc{Densest} $k$-\textsc{Subgraph} problem.


% other centralities
\subsection{Results for other centralities}

Waniek et al. next show that \HL is \NPc for the closeness \cite{Waniek2017} and betweenness \cite{Waniek2021full} centralities.
In more detail, they show that the problem is \Wh when parameterized by $b$ for both the closeness and the betweenness centralities.
In addition to this, the authors show that \HL cannot be approximated within a ratio of $(1 - \epsilon ) \cdot \ln(|F|)$
for any $\epsilon > 0$, unless \Po = \NP, with respect to both the closeness and the betweenness centralities.

For the core centrality measure, Dey and Medya \cite{Dey2019} next show that \HLshort is \NPc even if the core centrality of
every leader is exactly 3. Meaning that \HLshort is \pNPh when parameterized by the maximum core centrality of leaders.
In addition to this, the authors prove that \HLshort is polynomial-time solvable
if the core centrality of every leader is at most 1.
They complement that by proving that there does not exist any $((1 - \alpha) \cdot \ln(n))$-approximation algorithm
for any constant $\alpha \in (0, 1)$ which optimizes the number of edges that one needs to add even
when the core centrality of every leader is 3.
 