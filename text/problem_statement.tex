%---------------------------------------------------------------
\chapter{Problem statement}
%---------------------------------------------------------------

Now we define the central problem of this thesis, the \HL problem.

\begin{definition}[Hiding Leaders]\label{HL}
    Let $(G, L, b, c, d)$ be the problem instance, then
    \begin{description}
        \item $G = (V, E)$ is a network,
        \item $L \subseteq V$ are leaders and the remaining vertices $L = V \setminus L$ are followers,
        \item $c : G \times V \rightarrow \mathbb{R}$ is a centrality measure,
        \item $b$ is a maximum number of edges that we are allowed to add in $G$,
        \item $d$ is a safety margin -- a number of followers whose final centrality shoulds be at least as high as of any leader.
    \end{description}
    Given this instance, the goal is to identify a set of maximum $b$ edges between followers W $\subseteq$ F {\texttimes} F
    such that the resulting network $G' = (V, E \cup |W|)$ contains at least $d$ followers $F' \subseteq F$
    whose centrality must be at least as high as the centrality of any leader, that is
    $$|W| \leq b$$
    and
    $$\exists_{F' \subseteq F} |F'| \geq d \wedge \forall_{f \in F'} \forall_{l \in L} c(G', f) \geq c(G', l).$$
\end{definition}

There are two definitions of the \HL problem in the literature and they differ from each other in some details.
Namely, the definition from Waniek et al. \cite{Waniek2017} allows leaders $L$ to be equal to $V$, $L \subseteq V$,
which is a small difference, but the more important difference is that it requires centrality measure of followers $F'$ to be strictly greater
than that of any leader, $\forall_{f \in F'} \forall_{l \in L} c(G', f) > c(G', l)$.
We decided to stick to the definition from Dey \cite{Dey2019} because it seems more natural to us to let followers to be part of a solution
as soon as their centrality is at least the same as of all the leaders. If no leader has centrality measure greater than any follower from $F'$,
we consider the leaders hidden and safe from detection.
On the other hand, the name ``\HL'' is from the original definition by Waniek et al. as the definition tells us there can be many leaders in the network.
Later in section \ref{sec:proofRevs}, we show that complexity results from Waniek et al.
are correct even with this slighty different definition of \HLshort.
