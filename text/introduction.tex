%---------------------------------------------------------------
\chapter*{Introduction}\addcontentsline{toc}{chapter}{Introduction}\markboth{Introduction}{Introduction}
%---------------------------------------------------------------

\setcounter{page}{1}

Covert networks, or covert organizations, are social structures whose one of the main concerns is to operate in secret,
concealed from view of public, view of governments and intelligence agencies or view of any other unwanted subject.
Covert network analysis (CNA), as an important part of social network analysis (SNA),
is then a set of tools and techniques used to study covert networks and their members,
aiming to uncovering connections between the members and identifying those with a great
influence and thus importance, despite many connections remain unknown during this identification.
Such influential users are called leaders.
One of the most used techniques in the SNA toolset, and CNA in particular, for detecting leaders of covert networks
are \emph{centrality measures},
the concept introduced by Bavelas \cite{Bavelas1948} describing how important, or central,
a member of given network inside this network is.
Being crucial for their working, leaders try to stay undetected by such measures,
for which they use various counter-techniques such as
network modification by adding or deleting edges, or creating whole new networks from scratch.

In this thesis, we survey covert networks and hiding inside of them from a perspective of the \HL problem (\HLshort), first defined by
Waniek et al. \cite{Waniek2017} and further studied by various authors \cite{Dey2019,Waniek2021,Mohan2023}.
More concretely, our main concern is the \HL problem with respect to a degree centrality measure
and using the framework of parameterized computational complexity.
Because the problem is \NPh by itself, we expose it to various parameters and
see how these parameters affect its computational complexity.
Another goal of this thesis is to analyze the problem variants studied in the literature
and describe the current state of the research in the language of the parameterized complexity framework.
Profound description of the \HL problem can be found in Chapter \ref{chapter:ProblemStatement}, for introduction to
the theory of parameterized complexity, please refer to Section \ref{section:ParamComp}.
