%---------------------------------------------------------------
\chapter*{Introduction}\addcontentsline{toc}{chapter}{Introduction}\markboth{Introduction}{Introduction}
%---------------------------------------------------------------

\setcounter{page}{1}

Covert networks, or covert organizations, are social structures whose one of the main concerns is to operate in secret,
concealed from view of public, view of governments and national security agencies or view of any other unwanted subject.
Covert network analysis, as an important part of Social network analysis, is then a set of tools and techniques used to study
covert networks and their members, aiming to uncovering connections between the members and identifying those with a great
influence and thus importance, despite many connections remain unknown during this identification.
Such influential users are called leaders.
Being crucial for their working, leaders try to stay undetected for which they use various counter-techniques such as
network modification by adding or deleting edges, or creating whole new networks from scratch.

In this thesis, we survey covert networks from a perspective of the \HL problem first defined by
Waniek et al. \cite{Waniek2017} and further studied by various authors \cite{Dey2019,Waniek2021,Mohan2023}.
More concretely, our main concern is the \HL problem with respect to a degree centrality measure
and using the framework of parameterized complexity.
Profound description of the \HL problem can be found in Chapter \ref{chapter:ProblemStatement}, for introduction to
the theory of parameterized computational complexity, please refer to Section \ref{section:ParamComp}.


%---------------------------------------------------------------
\section{Literature review}

In this section, we bring a thorough examination of the literature on two topics;
social network analysis and covert networks, together with parameterized complexity.
The former is an essential knowedge providing us with better context for studying the \HL problem, whereas
the later gives us tools to better reason about \NPh problems which, as we will see, the \HL problem is.

\subsection{Social network analysis \& covert networks}

First of all, we mention the publication from Morselli \cite{Morselli2009} that the reader can refer to get
an extensive study of criminal networks.
Next, for recent and exhaustive overview of SNA techniques used for covert network disruption,
the reader can use the paper from Ficara et al. \cite{Ficara2022}.

Arguably the most extensive line of research is that in which authors use covert networks to analyze structures consisting of
harmful actors.
That can be terorist organizations or other structures interested in involving into illegal activities. These networks are
also called \emph{dark networks} \cite{Raab2003}.
Indeed, research in this direction have positive impact on understanding such harmful organisations,
shedding light on their inner processes and the most potentially hazardous actors and helping in dismantling the entire network,
preventing further operations and minimizing damage done.
The understanding of how a covert network works inside, can lead to understanding of how it can be broken.
Among many others, we mention work of these authors \cite{Waniek2017,Dey2019,Raab2003,Lindelauf2009,Xu2005,Ressler2006}.
However, as noted by Saavedra-Nieves and Casas-Méndez \cite{SaavedraNieves2023}, this view on covert networks as organizations with harmful
intends is not the only possible, since the usefulness of Covert network analysis lingers even in handling more global or day-to-day events
such as studying activist groups \cite{Crossley2012} or even sports teams \cite{Buldú2019}.

One of the most used techniques in the SNA toolset, and CNA in particular, are centrality measures,
the concept introduced by Bavelas \cite{Bavelas1948},
describing how important, or central, the member of given network inside this network is.
This is because finding the most important vertices within a given network is a natural approach when studying real-word human networks \cite{Crescenzi2016}.
    The most classical centrality measures are degree \cite{Shaw1954}, closeness \cite{Beauchamp1965},
    betweenness \cite{Anthonisse1971,Freeman1977} and core \cite{Seidman1983} centralities.
On top of the standard centrality measures, there are more recent and advanced techniques suitable for measuring one's rank within network,
for example the \emph{Game of Thieves} algorithm from Mocanu et al. \cite{Mocanu2018}, which can be used even within massive networks.
For the sake of completion, we also mention there exist centrality measures on edges, e.g.,
\emph{WERW-Kpath} algorithm from {De Meo} \cite{DeMeo2013}.
The two aforementioned algorithms are more reviewed by Ficara et al. \cite{Ficara2021}.

Apart from centrality measures, there are other important tools SNA has to offer.
One of them are node similarity measures which are with advantage used in link prediction \cite{Zhou2009,Wang2014}.
Link prediction methods also play an important part of research of covert networks
as they allow us to predict the existence of otherwise hidden connections between members, potentially leading to
better indentification of important members and relations between them.
Besides CNA, link prediction have other useful applications i.e. in recommendation systems \cite{Huang2005,Talasu2017}.

Next, we mention that leaders of covert networks often (but not always \cite{Fatih2012}) face what is called
\emph{efficiency/security trade-off} \cite{Morselli2007},
the dilemma between staying sufficiently hidden while keeping enough influence over the network,
which is the often-mentioned topic in the literature about covert networks \cite{Crossley2012,Waniek2017,Lindelauf2009}.
However, the ways in which this trade-off is addressed vary in the literature.
Waniek et al. \cite{Waniek2017} for example approach this by modeling security from the perspective of
centrality measures (centrality measure based secrecy) and efficiency from the perspective of \emph{models of influence}
and they show how to construct a network designed specifically to hide its leaders while
keeping their ability to influence the rest of the network.
Lindelauf et al. \cite{Lindelauf2009} on the other hand combine graph theory with game theory to model this dilemma.

When studying covert networks, we often talk about two parties, evaders and seekers;
evaders are members (often leaders) of a covert network;
seekers are then those who try to identify the evaders.
Members of the covert networks are typically aware of the importance of their network structure, as can be seen from
the efficiency/security dilemma.
However, when talking about seekers and evaders, authors typically consider these three scenarios:
(i) Evaders act unstrategically in the sense that they are unaware of
the seekers and SNA tools, centrality measures in particular, they use to reveal them;
(ii) Evaders behave as strategic actors, well-awared of the centrality analysis seekers use against them \cite{Waniek2017,Dey2019,Dey2020}.
(iii) Both seekers and evaders are strategic.
To this end, we mention the work from Waniek et al. \cite[y.~2021]{Waniek2021} in which the authors propose
the strategy of which centralities seekers should use to maximize the chances of detecting
a leader.

In the terminology of this section, the \HL problem describes a covert network consisting of strategic leaders, trying to
evade being detected by use of centrality measuring techniques,
for which they modify the network by adding new connections between network members. 

\subsection{Parameterized computational complexity}

What follows is a sumamry of the literature on the topic of parameterized complexity.
For an introduction to the topic itself, please refer to Section \ref{section:ParamComp}.

The foundations of parameterized complexity were laid by Downey and Fellows
in the series of papers from years 1992 to 1995 \cite{Downey1992,Downey1995.1,Downey1995.2,Downey1993,Downey1995.4},
which the authors further presented later in 1999 in their book \cite{Downey1999},
which was refined and once more published in 2013 \cite{Downey2013}.
Other relevant and potentially useful literature on this topic are
two books by Flum with Grohe \cite{Flum2006} and by Niedermeier \cite{Niedermeier2006} from 2006, book by Hans et al. \cite{Hans2012} from 2012,
book by Cygan et al. \cite{Cygan2015} from 2015 (which to a great extent covers knowledge from the previous literature)
and two books by Haan \cite{Haan2019} and by Fomin et al. (focused mainly on kernelization) \cite{Fomin2019} from 2019.
However, keep in mind that there is a highly active line of research in the field of parameterized complexity, so,
as researchers keep making new discoveries, information presented in the work of mentioned authors may not necessarily be up-to-date.
