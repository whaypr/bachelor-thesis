%---------------------------------------------------------------
\chapter{Our results}
%---------------------------------------------------------------

\begin{theorem}
    \HL parameterized by $b + d$ is W[1]-hard even if $|L| = 1$.
\end{theorem}

% Proof
\begin{proof}\label{proofDB}
    To proof this theorem, we propose a parameterized reduction from the $k$-\textsc{Clique} problem on regular graphs,
    which is known to be W[1]-hard with respect to $k$
    
    Suppose we have a parameterized $k$-\textsc{Clique} instance $(G, k)$, where $|V(G)|=n$, $G$ is a $r$-regular graph and $n-2 \geq r \geq k \geq 3$
    (for $r > n-2$ or $r < k$ or $k < 3$ the problem is trivial).
    Then, we construct a graph $H$ in the following way:
    \begin{enumerate}
        \item Start with an empty graph $H$.
        \item Add all vertices from $G$, i.e., $V(H) \leftarrow V(G)$, mark these vertices as $F_1$.
        \item Add all edges from $\overline{G}$, i.e., $E(H) \leftarrow E(\overline{G})$.
        \item Add an extra vertex $\ell$, i.e., $V(H) \leftarrow V(H) \cup \{\ell\}$.
        \item Add an edge between vertex $\ell$ and $n - r + (k - 1) \eqqcolon \lambda$ arbitrarily chosen vertices $X \subset F_1$,
              i.e., $E(H) \leftarrow E(H) \cup \{ (\ell, x) \mid x \in X \}$.
        \item For each vertex $v \in V(H) \setminus X \eqqcolon Y$, introduce a new vertex $w_v$ and add edge $\{v, w_v\}$, i.e.,
              $V(H) \leftarrow V(H) \cup \{ w_v \mid v \in Y \}$ and $E(H) \leftarrow E(H) \cup \{ (v, w_v) \mid v \in Y \}$,
              mark set of $w_v$ for each $v \in Y$ as $F_2$.
    \end{enumerate}
    An example of such construction can be seen in figure \ref{fig:proofDB}.

    Note that step 5 can always be done because $r \ge k$, so $|F_1| = n > n - r + (k - 1)$
    and since $deg(l) = 0$ (before step 5), there are enough vertices for $\ell$ to connect with.
    Considering $G$ is $r$-regular, its complement, constructed and added into $H$ in steps 2 and 3, must be $(n-r-1)$-regular.
    After connecting all the $F_1$ vertices either with $\ell$, or the corresponding $w_v$ in steps 5 and 6,
    they all end up with degree $n-r$;
    in other words, graph $H[F_1]$ is $(n-r)$-regular.
    Now we show how finding a $k$-clique in graph $G$ corresponds to finding a solution $W$ of \HLshort in $H$.

    Let us take $ \mathcal{I} = (H, \{\ell\}, \frac{k\cdot(k-1)}{2}, c, k)$, where $c$ is a degree centrality measure,
    as an instance of \HLshort with $k' \coloneqq b + d = \frac{k\cdot(k-1)}{2} + k$ as a parameter.
    First we note that degree of the only leader $\ell$ is $\lambda$, as presented above.
    The other vertices $V(H) - l$ are naturally followers of which we can recognize two types, $F_1$ and $F_2$,
    where $\forall_{f_1 \in F_1} deg(f_1)$ = $n-r$ and $\forall_{f_2 \in F_2} deg(f_2)$ = $1$.
    Whereas $F_1$ are vertices of the original graph $G$,
    $F_2$ plays a role of ``partners'' of vertices $Y$, since their only job is to substitute a missing connection with $\ell$.

    Because we just have one leader in $\mathcal{I}$, it is clear that for any follower $f' \in F'$ applies that
    $deg(f') \geq \lambda = n - r + (k - 1)$.
    Also, because $n > r$, $\max\limits_{\forall f_1 \in F_1}deg(f_1) = n-r > 1 = \max\limits_{\forall f_2 \in F_2}deg(f_2)$.
    With this in mind, we state the following:

    \begin{lemma}\label{lemmaInProof}
        No $F'$ can contain vertex from $F_2$, i.e., $F' \cap F_2 = \emptyset$.
    \end{lemma}
    \begin{subproof}
        Assume $F' \subseteq F_1$. This is a valid assumption because $|F_1| = n > k$ and it must be at least $d = k$ vertices in $F'$.
        Because all vertices from $F_1$ have degree $n-r$, we must have added at least $\lambda - (n - r) = k - 1$ new neighbours to all vertices from $F'$.
        Since the budget $b$ in $\mathcal{I}$ is $\frac{k\cdot(k-1)}{2}$,
        the only possible way of how $F'$ could have been constructed is that $|F'|=k$ and the newly added edges $W \subset F_1 \times F_1$, $|W| = b$
        connect all the $k$ vertices $f' \in F'$ in a way that graph $(F', W)$ forms a $k$-clique.
        From this and from the statements above, we can see that no vertex from $F_2$ can be in $F'$
        since there will never be large enough budget $b$ to let us reach the safety margin $d$.
    \end{subproof}

    As shown in the proof of lemma \ref{lemmaInProof}, every feasible solution $W$ of \HLshort for instance $\mathcal{I}$, together with corresponding $F'$,
    will form a $k$-clique.
    Note that every edge from $W$ must be an edge from the original graph $G$ because $W \subset F_1 \times F_1$ and $F_1 = E(G)$
    and also all the edges from $E(\overline{G})$ are already in $E(H)$ so they cannot be in $W$ since it only contains newly added edges.

    Because a solution $W$ for \HLshort in $H$ exists if and only if a $k$-clique in $G$ exists,
    we can conclude that finding a solution for the \HL problem in graph $H$ is exactly the same as
    finding a solution for the $k$-\textsc{Clique} problem in graph $G$.

    The reduction presented in this proof is a valid parameterized reduction because;
    $(G, k)$ is a yes-instance of the $k$-\textsc{Clique} problem if and only if $\mathcal{I}$ is a yes-instance of \HLshort;
    the construction of $H$ is done in time polynomial to $n$;
    and there is a function of $k$, $g(k) = \frac{k\cdot(k-1)}{2} + k$, upper-bounding the parameter $k'$ of $\mathcal{I}$.
\end{proof}
